\clearpage
\pagestyle{empty}

\begin{center}
	\smallskip

	\Large \bfseries \MakeUppercase{
		Lembar Identitas \\
		Tugas Akhir Capstone
	}
	\vspace{0.5cm}

	\raggedright
	\begin{table}[h!]
		\large \bfseries
		\begin{onehalfspace}
		\begin{tabular}{p{0.3\textwidth} p{0.63\textwidth}}
			Judul Proyek TA : & \capstonetitle \\
		\end{tabular}
		\end{onehalfspace}
	\end{table}

	\normalsize \normalfont

	Anggota tim dan pembagian peran:

	\begin{table}[h!]
		\begin{onehalfspace}
		\begingroup
		\def\arraystretch{1.25}
		\begin{tabular}{|p{0.05\textwidth} | p{0.13\textwidth} | p{0.19\textwidth} | p{0.50\textwidth}|}
			\hline
			\textbf{No.} & \textbf{NIM} & \textbf{Nama}         & \textbf{Peran}                                                                                                 \\
			\hline
			1.           & 13519116     & Jeane Mikha Erwansyah & Pengimplementasian Objek Lokal dan Lingkungan Indonesia untuk Simulasi Trem Otonom Menggunakan Simulator CARLA \\
			\hline
			2.           & 13519164     & Josep Marcello        & Pembuatan Jalur Komunikasi Antara Simulator CARLA dengan Server NVIDIA Pegasus                                 \\
			\hline
			3.           & 13519188     & Jeremia Axel Bachtera & Pembangunan Skenario Pengujian Simulasi Trem Otonom di Indonesia                                               \\
			\hline
		\end{tabular}
		\endgroup
		\end{onehalfspace}
	\end{table}

	\vfill
	\begin{center}
		\normalsize \normalfont
		Bandung, \thedate \\
		Mengetahui,
	\end{center}
	\advisorapproval

\end{center}
\clearpage
