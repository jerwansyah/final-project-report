\chapter{Studi Literatur}

Bab ini membahas hasil studi literatur yang diperlukan untuk melakukan
penyusunan Tugas Akhir ini. Bab ini membahas tentang kendaraan otonom, simulator
CARLA, Blender, dan beberapa penelitian terkait.
%RoadRunner,

\section{Kendaraan Otonom}

Kendaraan otonom atau \textit{Autonomous Vehicle} merupakan kendaraan yang
dirancang untuk memonitor keadaan jalan dan berkemudi sendiri tanpa bantuan
pengemudi. Kendaraan otonom dibuat dengan tujuan mengurangi jumlah kecelakaan,
mengurangi penggunaan energi, mengurangi polusi, mengurangi kemacetan, dan
meningkatkan akses transportasi \parencite{av-bagloee}. Trem otonom sangat mirip
dengan mobil otonom, keduanya beroperasi di lingkungan yang sama dan
berinteraksi dengan objek-objek yang sama pula. Perbedaan keduanya adalah trem
otonom terikat dengan rel trem sehingga trem tidak dapat berbelok keluar dari
rel untuk menghindari objek yang berpotensi menjadi halangan ataupun berhenti
mendadak karena terbatasi secara fisik dan berhenti mendadak dapat mencelakai
penumpang \parencite{at-palmer}.

Trem otonom secara umum memiliki perangkat keras berupa kendaraan fisik,
komputer, dan berbagai sensor. Komputer pada trem digunakan untuk mengoperasikan
trem secara otonom. Komputer tersebut memiliki kartu grafis untuk melakukan
deteksi objek hasil deteksi dari sensor-sensor di trem. Sensor-sensor tersebut
adalah, kamera, LIDAR, dan radar. Pada sisi perangkat lunak, trem otonom pada
umumnya memiliki aplikasi untuk mengolah data sensor \parencite{at-palmer}.

Model pemilihan keputusan oleh kendaraan otonom perlu dilatih sehingga kendaraan
otonom tersebut dapat diandalkan. Oleh karena itu, diperlukan penelitian dan
percobaan lebih lanjut untuk memngembangkan algoritma, kemampuan evaluasi, dan
protokol kendaraan otonom. Penelitian dan eksperimen kendaraan otonom dapat
dilakukan dengan melakukan simulasi detail yang banyak. Simulasi-simulasi
tersebut bergantung kepada model yang ada di dunia nyata \parencite{av-berger}.

Simulasi kendaraan otonom akan memudahkan pelatihan dan proses validasi strategi
kemudi kendaraan otonom. Simulasi mencegah terjadinya kejadian yang berbahaya
dan kejadian yang tidak diinginkan terjadi, seperti penelitian fisik. Penelitian
kendaraan otonom di dunia nyata memiliki banyak rintangan. Penelitian fisik
membutuhkan biaya yang besar untuk membangun kendaraan otonom tersebut,
infrastruktur untuk pengetesan yang memadai, dan biaya lainnya. Selain
dibutuhkannya biaya, dibutuhkan juga sumber daya manusia untuk mengoperasikan
tes. Pengumpulan data juga membutuhkan waktu yang lama karena dibutuhkannya data
yang banyak, data yang valid, dan data yang mencakup kasus-kasus tidak terduga
untuk melatih kendaraan otonom \parencite{carla-dosovitskiy}.

% note:
% about tram specifically
% autonomous vehicles \subsection{Kendaraan Otonom di Indonesia}
% \subsection{Trem Otonom di Indonesia}
% \subsection{Lingkungan Berkendara di Indonesia ?}

% \section{Lingkungan Berkendara di Indonesia}
% street condition, layout, surrounding objects
% \section{Peraturan Lalu Lintas Indonesia}
% traffic rules in Indonesia

\section{Simulator CARLA}

\textit{Car Learning to Act} atau CARLA merupakan sebuah perangkat lunak
\textit{open-source} yang berguna untuk melakukan simulasi kendaraan otonom dan
lingkungan urban. CARLA telah dikembangkan untuk mendukung pelatihan, pembuatan
purwarupa, dan pengujian strategi kemudi otonom, baik dari sisi persepsi maupun
sisi pengontrolan. CARLA menyediakan model dasar untuk lingkungan simulasi,
sejumlah kendaraan, rambu lalu lintas, dan pejalan kaki. Selain itu, CARLA juga
menyediakan keluaran sensor-sensor dan sinyal-sinyal yang dapat digunakan untuk
melatih strategi kemudi, seperti koordinat GPS, kecepatan, percepatan, data
detail mengenai tabrakan yang terjadi pada model kendaraan, dan lain-lain.
Sensor-sensor dan kondisi lingkungan simulasi tersebut dapat diatur sesuai
dengan kebutuhan \parencite{carla-dosovitskiy}.

% \section{\textit{CARLA Simulation Engine}}
CARLA dikembangkan sedemikian rupa sehingga fleksibel untuk disesuaikan dengan
mudah dan juga realistis secara visual dan secara fisika. CARLA diimplementasi
sebagai layar \textit{open-source} di atas Unreal Engine 4 (UE4) sehingga dapat
dikembangkan lebih lanjut oleh komunitas \textit{open-source}.
\textit{Simulation engine} tersebut menyediakan kualitas \textit{render} yang
terkemuka, hukum fisika yang realistis, logika \textit{Non-Playable Character}
(NPC) dasar, dan banyak \textit{plugin} lainnya \parencite{carla-dosovitskiy}.

CARLA menyimulasikan sebuah dunia yang dinamis dan juga menyediakan antarmuka
sederhana antara dunia tersebut dan aktor yang berinteraksi dengan dunia
tersebut. CARLA dirancang sebagai sistem \textit{server-client} sehingga
fungsionalitas tersebut dapat direalisasikan. Server CARLA menjalankan dan
menyimulasikan suasana. \textit{Client} bertanggung jawab untuk mengatur
interaksi antara aktor dengan server via \textit{socket} dengan mengirimkan
perintah kepada aktor dan perintah untuk mengatur peraturan lingkungan simulasi
di server. \textit{Client} akan menerima umpan balik hasil rekaman sensor
\parencite{carla-dosovitskiy}.

Lingkungan simulasi pada CARLA terdiri atas model-model 3 dimensi (3D) yang
statik, seperti bangunan, tumbuh-tumbuhan, rambu lalu lintas, dan infrastuktur
lainnya. Selain itu, terdapat juga model-model dinamis, seperti kendaraan dan
pejalan kaki. Model-model 3D tersebut telah dibuat sehingga dimensi model-model
tersebut mencerminkan dimensi objek aslinya di dunia nyata. Model-model tersebut
dibuat menggunakan model geometrik yang ringan dan dengan tekstur yang sesuai
sehingga terlihat realistis dan detail serta dapat di-\textit{render} dengan
cepat. Aset-aset yang telah disediakan CARLA sangat dapat disesuaikan
\parencite{carla-dosovitskiy}. Aset baru juga dapat ditambahkan
\parencite{carla-documentation}.

\subsection{Unreal Engine}

Unreal Engine merupakan sebuah \textit{game engine} yang dikembangkan oleh Epic
Games. Unreal Engine merupakan sebuah perangkat lunak yang dapat digunakan untuk
membuat permainan, simulasi, dan aplikasi multimedia lainnya. Unreal Engine
memiliki fitur-fitur untuk membuat konten tiga dimensi yang sangat lengkap
\parencite{ue-5}. Jika fitur atau ekosistem Unreal Engine kurang sesuai dengan
kebutuhan pengembang, Unreal Engine menyediakan \textit{source code} yang dapat
dimodifikasi \parencite{ue-4}. CARLA menggunakan Unreal Engine untuk
pengembangan lingkungan simulasi \parencite{carla-documentation}.

\subsection{Konsep Dasar CARLA \parencite{carla-documentation}}

CARLA memiliki beberapa konsep dasar sebagai berikut:

\begin{enumerate}

    \item \textit{World} dan \textit{Client}
    \item \textit{Actor} dan \textit{Blueprint}
    \item Map dan Navigasi
    \item Sensor dan Data

\end{enumerate}

% \subsubsection{\textit{World} dan \textit{Client}}

\textit{Client} dari CARLA merupakan modul yang dijalankan pengguna untuk
mendapatkan informasi atau untuk mengubah simulasi. Pengguna melakukan kedua hal
tersebut melalui sebuah terminal yang terhubung dengan server simulasi.
\textit{World} merupakan sebuah objek yang merepresentasikan simulasi. Objek
\textit{world} merupakan sebuah abstraksi kumpulan layar yang menyimpan berbagai
status dari dunia simulasi, memiliki metode untuk \textit{spawn} aktor,
melakukan pengontrolan cuaca, dan lain-lain.

% \subsubsection{\textit{Actor} dan \textit{Blueprint}}

Aktor (\textit{actor}) merupakan objek yang memiliki peran dalam simulasi. Aktor
pada CARLA berupa: kendaraan, pejalan kaki, sensor, pengamat, rambu lalu lintas,
dan lampu lalu lintas. \textit{Blueprint} berfungsi untuk memunculkan atau
melakukan \textit{spawn} aktor lain dengan tujuan mengintegrasikan aktor
tersebut ke dalam simulasi.

% \subsubsection{Map dan Navigasi}

Map atau peta merupakan objek yang merepresentasikan dunia atau lingkungan
simulasi. Peta pada CARLA direpresentasikan dalam bentuk model 3D dan perincian
jalan, persimpangan, dan jalur yang ada. CARLA menggunakan OpenDRIVE 1.4 sebagai
standar definisi jalan. Navigasi pada CARLA dibantu dengan objek rambu lalu
lintas dan lampu lalu lintas yang memiliki perincian informasi OpenDRIVE. Kedua
objek tersebut membentuk balok pembatas \textit{bounding boxes} sehingga
aktor-aktor mengetahui bagaimana cara navigasi.

% \subsubsection{Sensor dan Data}

Sensor merupakan aktor yang dapat mengumpulkan data dari lingkungan sekitar.
Pengumpulan data yan efisien dilakukan dengan memasang sensor pada kendaraan
yang dikendalikan oleh pengguna dalam simulasi atau disebut sebagai \textit{ego
vehicle}.

% note: reference of why carla build version is needed for advaced customization
% and development: https://carla.readthedocs.io/en/0.9.13/start_quickstart/

\section{Blender \parencite{blender-manual}}

Blender adalah sebuah perangkat lunak \textit{open-source} untuk pembuatan model
3D. Blender dapat digunakan di berbagai sistem operasi dan berjalan lancar pada
Linux, Windows, dan Macintosh. Blender menyediakan berbagai fitur yang mendukung
keseluruhan \textit{pipeline} model 3D. Fitur-fitur tersebut adalah pembuatan
model, \textit{rigging}, pembuatan animasi, pembuatan simulasi,
\textit{rendering}, pembuatan komposit, pelacakan gerakan (\textit{motion
tracking}), pengeditan video, dan pembuatan permainan \parencite{blender-about}.
% Blender menyediakan API yang dapat digunakan untuk penyesuaian Blender sendiri
% dan pembuatan alat pengeditan khusus. API Blender dapat digunakan menggunakan
% bahasa pemrograman Python

\subsection{Mode Objek}

Blender memiliki berbagai mode objek yang berfungsi untuk mengedit berbagai
aspek dari objek. Beberapa mode objek blender adalah sebagai berikut:

\begin{enumerate}

    \item \textit{Object Mode}: Mode \textit{default} objek yang berguna untuk
    mengedit posisi, rotasi, skala objek, dan lain-lain.

    \item \textit{Edit Mode}: Mode untuk mengedit bentuk objek, seperti titik,
    sisi, permukaan, dan lain-lain.

    \item \textit{Texture Paint Mode}: Mode untuk menambahkan tekstur langsung
    model 3D.

    \item \textit{Pose Mode}: Mode untuk mengedit pose \textit{armature}.

\end{enumerate}

\subsection{Animasi dan \textit{Rigging}}

Animasi adalah membuat sebuah objek bergerak atau berubah bentuk dari waktu ke
waktu. \textit{Rigging} adalah istilah untuk menambahkan kontrol ke sebuah objek
dengan tujuan untuk menganimasikan objek tersebut. \textit{Rigging} pada umumnya
melibatkan komponen \textit{armature} dan komponen \textit{constraints}
(komponen pembatas). \textit{Armature} berguna agar objek memiliki sendi yang
fleksibel dan sering digunakan untuk animasi kerangka.

\subsection{\textit{Armature}}

\textit{Armature} pada Blender merupakan objek yang serupa dengan sistem
kerangka manusia terutama fungsionalitasnya. \textit{Armature} terdiri atas
kumpulan tulang atau \textit{bone} yang memiliki struktur/hierarki dan berfungsi
untuk memberi gerakan ke sebuah objek/model 3D. \textit{Armature} didesain untuk
diberi pose. Gambar \ref{fig:basic-armature} menunjukkan contoh struktur
\textit{armature} sederhana pada Blender. Kumpulan \textit{bone} pada sebuah
\textit{armature} tidak perlu berhubungan satu sama lain sehingga struktur dari
\textit{armature} dapat berupa-rupa, seperti rantai tulang atau \textit{chains
of bones} yang dapat dilihat pada Gambar \ref{fig:chains-of-bones}. Setelah
pembuatan \textit{armature} selesai, perlu dilakukan \textit{skinning} agar
pergerakan dari \textit{armature} berdampak pada objek lain. Proses
\textit{skinning} merupakan penghubungan antara objek \textit{armature} dengan
sebuah objek atau objek model/\textit{mesh}. Gambar \ref{fig:basic-armature}
juga menunjukkan \textit{armature} yang telah di-\textit{skinning}.

\begin{figure}[ht]
    \centering
    \includegraphics[width=0.28\textwidth]{resources/chapter-2-basic-armature.png}
    \caption{\textit{Skinned armature} sederhana \parencite{blender-manual}}
    \label{fig:basic-armature}
\end{figure}

\begin{figure}[ht]
    \centering
    \includegraphics[width=0.48\textwidth]{resources/chapter-2-chain-of-bones.png}
    \caption{\textit{Chains of Bones} \parencite{blender-manual}}
    \label{fig:chains-of-bones}
\end{figure}

\subsection{\textit{Bone}}

\textit{Bone} merupakan komponen dari \textit{armature}. Struktur sebuah
\textit{bone} dapat dilihat pada Gambar \ref{fig:bone-structure}. Sebuah
\textit{bone} terdiri atas 3 subkomponen adalah sebagai berikut:

\begin{enumerate}

    \item Sendi awal (\textit{root}/\textit{head})

    Sendi awal memiliki koordinat di sumbu X, Y, dan Z di ruang lokal objek
    \textit{armature}. Sendi awal merupakan titik acuan untuk rotasi
    \textit{bone} pada sumbu X dan Z pada \textit{local space}. Rotasi ini
    disebut juga sebagai \textit{roll angle}.

    \item Sendi akhir (\textit{tip}/\textit{tail})

    Sendi akhir memiliki koordinat pada sumbu X, Y, dan Z relatif terhadap sendi
    awal.

    \item Badan (\textit{body})

    Badan oktahedron dari \textit{bone} terbentuk untuk menghubungkan sendi awal
    dan sendi akhir. Bagian sudut oktahedron yang lebih besar berhubungan dengan
    sendi awal. Sebaliknya, bagian sudut oktahedron yang lebih kecil berhubungan
    dengan sendi akhir. Badan \textit{bone} menentukan arah sumbu Y lokal dan
    rotasi \textit{bone} objek ketika diposekan.

\end{enumerate}

\begin{figure}[ht]
    \centering
    \includegraphics[width=0.3\textwidth]{resources/chapter-2-bone-structure.png}
    \caption{Struktur \textit{bone} \parencite{blender-manual}}
    \label{fig:bone-structure}
\end{figure}

% \section{RoadRunner \parencite{roadrunner}}

% RoadRunner merupakan sebuah \textit{interactive editor} yang digunakan untuk
% membuat konten desain 3D untuk simulasi dan pengujian sistem kendaraan otonom.
% Pengguna dapat membuat desain jalanan dengan mengedit lingkungan dan menambahkan
% rambu lalu lintas, sinyal lalu lintas, persimpangan, dan batas kendaraan. Selain
% itu, RoadRunner dapat digunakan untuk membuat desain kota.

\section{Penelitian Terkait}

Subbab ini membahas beberapa penelitian terkait dengan Tugas Akhir ini.
Penelitian-penelitian berikut menjadi rujukan dalam penelitian dan pengembangan
Tugas Akhir ini.

\subsection{Pengembangan Sistem Otonomi dengan Menggunakan Kecerdasan \textit{Artificial} untuk Trem \parencite{rispro-trilaksono}}
\label{subsec:rispro-trilaksono}

Penelitian ini merupakan penelitian mengenai inovasi bidang otomotif yang
mengembangkan sistem otonomi untuk trem menggunakan kecerdasan buatan.
Penelitian tersebut memiliki tiga tahap, yaitu pengembangan \textit{tram driving
assistance}, pengembangan trem otonom, dan pengujian trem otonom di
\textit{mixed traffic} serta persiapan komersialisasi.

Berikut adalah Indikator Kinerja Riset (IKR) atau luaran dari penelitian
tersebut:

\begin{enumerate}

    \item Pengembangan algoritma persepsi untuk mengenali objek dan
    \textit{tracking} di lingkungan trem pada cuaca normal dan implementasi
    dalam bentuk \textit{software}.

    % Algoritma persepsi yang dikembangkan berupa \textit{panoptic segmentation},
    % \textit{camera object detection}, \textit{LIDAR object detection},
    % \textit{camera-LIDAR fusion object detection}, dan \textit{camera-radar
    % fusion object detection}

    \item Pengembangan algoritma perencanaan jalur untuk \textit{decision
    making} kendali kecepatan trem dan implementasi dalam bentuk
    \textit{software}.

    % Algoritma persepsi yang dikembangkan berupa \textit{railway estimator},
    % \textit{trajectory prediction}, dan \textit{safety assessment}

    \item Pengambilan \textit{dataset} lingkungan trem.

    \textit{Dataset} yang diambil bentuk gambar citra kamera biasa, gambar citra
    LIDAR, dan data lainnya kemudian diolah dengan penambahan anotasi
    seperlunya.

    \item Pengembangan \textit{Driving Assistance System} untuk trem berupa:
    \textit{object detection \& collision-avoidance assist}, \textit{speed limit
    assist}, dan \textit{face recognition \& driver attention warning}.

    \item Pengujian, analisis, dan desain ulang algoritma yang dikembangkan pada
    \textit{Software-in-the-Loop Simulation} (SILS) dan
    \textit{Hardware-in-the-Loop Simulation} (HILS).

    Dilakukan pengujian dan penganalisisan \textit{platform} simulasi simulator
    CARLA versi 0.9.12, uji coba beragam sensor, dan uji coba beberapa algoritma
    yang sudah dikembangkan. Dilakukan konversi kode program/algoritma yang
    telah dikembangkan ke bahasa pemrograman C++ untuk dimasukkan ke NVIDIA
    DRIVE AGX Pegasus (selanjutnya akan disebut sebagai Pegasus) dan membuat
    \textit{web-service} untuk menghubungkan server simulasi dengan Pegasus.
    Pegasus merupakan \textit{hardware} untuk memroses algoritma
    \textit{Adaptive Cruise Control} (ACC), \textit{Emergency Braking System}
    (EBS), dan \textit{Collision Avoidance} (\textit{decision making}). Dari
    pengujian dan analisis tersebut, dibutuhkan penyempurnaan algoritma persepsi
    berbasis sensor, perbaikan arsitektur (HILS), dan integrasi objek lokal pada
    skenario simulasi.

    \item Pengembangan dan manufaktur \textit{platform} trem, sistem
    \textit{drive-by-wire} pada trem, dan integrasi sensor.

    \item Publikasi ilmiah.
    \item Draf kekayaan intelektual.
    \item \textit{Self-assessment} Tingkat Kesiapan Teknologi (TKT).
    \item Penyusunan poster ilmiah populer atas pelaksanaan dan hasil riset.

\end{enumerate}

\subsection{\textit{KIT Bus: A Shuttle Model for CARLA Simulator} \parencite{related-work-xiang}}
\label{subsec:kitbus}

Penelitian \textit{KIT Bus: A Shuttle Model for CARLA Simulator} ini merupakan
penelitian membuat model \textit{shuttle bus}. Proses pembuatan model tersebut
dilakukan dalam tiga tahap, yaitu sebagai berikut:

\begin{enumerate}

    \item Membuat model 3D dari bus menggunakan aplikasi 3ds Max.

    Model bus dibuat sesuai dengan referensi dan dimensi asli. Model bus yang
    dibuat secara detail dan dengan permukaan mulus sehingga realistis. Model
    bus terdiri atas 6 bagian, yaitu bagian badan, bagian roda-roda, bagian
    interior, bagian detail, bagian kaca, dan bagian pelat kendaraan.

    \item Melakukan pengeditan model 3D bus dalam editor CARLAUE4.

    Model bus yang telah dibuat kemudian diimpor ke dalam editor CARLAUE4.
    Masing-masing bagian badan bus dan keempat roda bus dipasangkan sebuah
    \textit{collision box} yang berbentuk dan berukuran sama. Agar model bus
    dapat menyimulasikan animasi bus yang sebenarnya, sejumlah
    \textit{blueprint} dibuat sesuai dengan bagian bus yang dapat bergerak.
    Dilakukan juga penambahan material atau tekstur dan fungsionalitas lainnya
    yang dibutuhkan seperti, lampu dan tekstur-tekstur yang ada pada permukaan
    bus. Hal tersebut dilakukan sehingga model bus dapat menyerupai model bus
    yang sebenarnya. Model bus yang telah selesai diedit dimasukkan ke dalam
    \textit{vehicle factory} agar dapat bisa dimunculkan (\textit{spawn}).

    \item Memverifikasi kontrol manual dan otonom dari model 3D bus yang telah
    dibangun dalam CARLA.

    Model bus yang telah dibuat kemudian diuji untuk memverifikasi apakah model
    bus tersebut dapat dioperasikan secara manual dan otonom. Model bus dites
    untuk berjalan maju, mengerem untuk melambat, mengemudi ke kiri dan ke
    kanan, dan memindahkan gigi kopling.

\end{enumerate}

\subsection{\textit{The Autonomous Siemens Tram} \parencite{at-palmer}}

Penelitian \textit{The Autonomous Siemens Tram} membahas trem otonom Siemens
yang telah didemonstrasikan di Potsdam, Jerman pada tahun 2018. Sistem trem
otonom tersebut dibangun di atas trem Siemens Combino dan menggunakan
sensor-sensor multi-modal untuk mengidentifikasi lokasi kendaraan serta
mendeteksi dan merespons lampu lalu lintas dan objek lain. Trem otonom memiliki
beberapa komputer yang memiliki \textit{Graphics Processing Unit} (GPU) yang
memadai untuk memroses data dari sensor LIDAR, sensor radar, kamera pendeteksi
objek, dan kamera pendekteksi sinyal. Trem otonom tersebut dapat beroperasi
dengan lancar namun memiliki kendala ketika melakukan lokalisasi trem hanya
dengan \textit{Global Navigation Satellite System} (GNSS). Penelitian masih
dilakukan untuk mengatasi kendala tersebut, misalnya dengan melakukan lokalisasi
dengan persepsi.
