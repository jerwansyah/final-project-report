\chapter{Studi Literatur}

Bab ini membahas hasil studi literatur yang diperlukan untuk melakukan
penyusunan Tugas Akhir ini. Bab ini membahas tentang kendaraan otonom, simulator
CARLA, Unreal Engine, Blender, RoadRunner, dan penelitian terkait.

\section{Kendaraan Otonom}
Kendaraan otonom atau \textit{Autonomous Vehicle} merupakan kendaraan yang
dirancang untuk memonitor keadaan jalan dan berkemudi sendiri tanpa bantuan
pengemudi. Kendaraan otonom dibuat dengan tujuan mengurangi jumlah kecelakaan,
mengurangi penggunaan energi, mengurangi polusi, mengurangi kemacetan, dan
meningkatkan akses transportasi \parencite{av-bagloee}. Trem otonom sangat mirip
dengan mobil otonom, keduanya beroperasi di lingkungan yang sama dan
berinteraksi dengan objek-objek yang sama pula. Perbedaan keduanya adalah trem
otonom terikat dengan rel trem sehingga trem tidak dapat berbelok keluar dari
rel untuk menghindari objek yang berpotensi menjadi halangan ataupun berhenti
mendadak karena terbatasi secara fisik dan berhenti mendadak dapat mencelakai
penumpang \parencite{at-palmer}.

Trem otonom secara umum memiliki perangkat keras berupa kendaraan fisik,
komputer, dan berbagai sensor. Komputer pada tram digunakan untuk mengoperasikan
trem secara otonom. Komputer juga memiliki kartu grafis untuk melakukan deteksi
objek hasil deteksi dari sensor-sensor di trem. Sensor-sensor tersebut adalah,
kamera, lidar, dan radar. Sedangkan pada sisi perangkat lunak, trem otonom pada
umumnya memiliki aplikasi untuk mengolah data sensor \parencite{at-palmer}.

Model pemilihan keputusan oleh kendaraan otonom perlu dilatih sehingga kendaraan
otonom tersebut dapat diandalkan. Oleh karena itu, diperlukan penelitian dan
percobaan lebih lanjut untuk memngembangkan algoritma, kemampuan evaluasi, dan
protokol kendaraan otonom. Penelitan dan eksperimen kendaraan otonom dapat
dilakukan dengan melakukan simulasi detil yang banyak. Simulasi-simulasi
tersebut bergantung kepada model yang ada di dunia nyata \parencite{av-berger}.

Simulasi kendaraan otonom akan memudahkan pelatihan dan proses validasi strategi
kemudi kendaraan otonom. Simulasi mencegah terjadinya kejadian yang berbahaya
dan tidak diinginkan terjadi seperti jika penelitian fisik dilakukan. Penelitian
kendaraan otonom di dunia nyata memiliki banyak rintangan. Penelitian fisik
membutuhkan biaya yang besar untuk membangun kendaraan otonom tersebut,
infrastuktur untuk pengetesan yang memadai, dan biaya lainnya. Selain
dibutuhkannya biaya, dibutuhkan juga sumber daya manusia untuk mengoperasikan
tes. Pengumpulan data juga membutuhkan waktu yang lama karena dibutuhkannya data
yang banyak, data yang valid, dan data yang mencakup kasus-kasus tidak terduga
untuk melatih kendaraan otonom \parencite{carla-dosovitskiy}.

% TODO
% autonomous vehicles \subsection{Kendaraan Otonom di Indonesia}
% \subsection{Trem Otonom di Indonesia}
% \subsection{Lingkungan Berkendara di Indonesia ?}

% \section{Lingkungan Berkendara di Indonesia}
% street condition, layout, surrounding objects
% \section{Peraturan Lalu Lintas Indonesia}
% traffic rules in Indonesia

\section{Simulator CARLA}
\textit{Car Learning to Act} atau CARLA merupakan sebuah perangkat lunak
\textit{open-source} yang berguna untuk melakukan simulasi kendaraan otonom dan
lingkungan urban. CARLA telah dikembangkan untuk mendukung pelatihan, pembuatan
purwarupa, dan pengujian strategi kemudi otonom, baik dari sisi persepsi maupun
sisi pengontrolan. CARLA menyediakan model dasar untuk lingkungan simulasi,
sejumlah kendaraan, rambu lalu lintas, dan pejalan kaki. Selain itu, CARLA juga
menyediakan keluaran sensor-sensor dan sinyal-sinyal yang dapat digunakan untuk
melatih strategi kemudi seperti koordinat GPS, kecepatan, percepatan, data rinci
mengenai tabrakan yang terjadi pada model kendaraan, dan lain-lain.
Sensor-sensor dan kondisi lingkungan simulasi tersebut dapat diatur sesuai
dengan kebutuhan \parencite{carla-dosovitskiy}.

% \section{\textit{CARLA Simulation Engine}}
CARLA dikembangkan sedemikan rupa sehingga fleksibel untuk disesuaikan dengan
mudah dan juga realistis secara visual dan secara fisika. CARLA diimplementasi
sebagai layar \textit{open-source} di atas Unreal Engine 4 (UE4) sehingga dapat
dikembangkan lebih lanjut oleh komunitas \textit{open-source}.
\textit{Simulation engine} tersebut menyediakan kualitas \textit{render} yang
terkemuka, hukum fisika yang realistis, logika \textit{Non-Playable Character}
(NPC) dasar, dan banyak \textit{plugin} \parencite{carla-dosovitskiy}.

CARLA menyimulasikan sebuah dunia yang dinamis dan juga menyediakan antarmuka
sederhana antara dunia tersebut dan aktor atau agen yang berinteraksi dengan
dunia tersebut. CARLA dirancang sebagai sistem \textit{server-client} sehingga
fungsionalitas tersebut dapat direalisasikan. Server CARLA menjalankan dan
menyimulasikan suasana. Klien bertanggung jawab untuk mengatur interaksi antara
aktor atau agen dengan server via \textit{socket} dengan mengirimkan perintah
untuk aktor dan perintah untuk mengatur peraturan lingkungan simulasi di server.
Klien akan menerima umpan balik hasil rekaman sensor
\parencite{carla-dosovitskiy}.

Lingkungan simulasi pada CARLA terdiri atas model-model 3 dimensi (3D) yang
statik seperti bangungan, tumbuh-tumbuhan, rambu lalu lintas, dan infrastuktur
lainnya. Selain itu, terdapat juga model-model dinamis seperti kendaraan dan
pejalan kaki. Model-model 3D tersebut telah dibuat sehingga dimensi model-model
tersebut mencerminkan dimensi objek aslinya di dunia nyata. Model-model tersebut
dibuat menggunakan model geometrik yang ringan dan dengan tekstur yang sesuai
sehingga terlihat realistis dan detil serta dapat di-\textit{render} dengan
cepat. Aset-aset yang telah disediakan CARLA sangat dapat disesuaikan
\parencite{carla-dosovitskiy}. Aset baru juga dapat ditambahkan
\parencite{carla-documentation-intro}.

\section{Unreal Engine}
Unreal Engine merupakan sebuah \textit{game engine} yang dikembangkan oleh Epic
Games. Unreal Engine merupakan sebuah perangkat lunak yang dapat digunakan untuk
membuat permainan, simulasi, dan aplikasi multimedia lainnya. Unreal Engine
memiliki fitur-fitur untuk membuat konten tiga dimensi yang sangat lengkap
\parencite{ue-5}. Jika fitur atau ekosistem Unreal Engine kurang sesuai dengan
kebutuhan pengembang, Unreal Engine menyediakan \textit{source code} yang dapat
dimodifikasi \parencite{ue-4}. CARLA menggunakan Unreal Engine untuk
pengembangan lingkungan simulasi \parencite{carla-documentation-build}.
% TODO about unreal engine 4 about carla unreal engine (+editor)
% maybe how to add objects?

\section{Blender}
Blender adalah sebuah perangkat lunak \textit{open-source} untuk pembuatan model
3D. Blender dapat digunakan di berbagai sistem operasi dan berjalan lancar pada
Linux, Windows, dan Macintosh. Blender menyediakan berbagai fitur yang mendukung
keseluruhan \textit{pipeline} model 3D. Fitur-fitur tersebut adalah pembuatan
model, \textit{rigging}, pembuatan animasi, pembuatan simulasi,
\textit{rendering}, pembuatan komposit, pelacakan gerakan (\textit{motion
tracking}), pengeditan video, dan pembuatan permainan. Blender menyediakan API
yang dapat digunakan untuk penyesuaian Blender sendiri dan pembuatan alat
pengeditan khusus. API Blender dapat digunakan menggunakan bahasa pemrograman
Python \parencite{blender-about}.

\section{RoadRunner}
RoadRunner merupakan sebuah \textit{interactive editor} yang digunakan untuk
membuat konten desain 3D untuk simulasi dan pengujian sistem kendaraan otomatis.
Pengguna dapat membuat desain jalanan dengan mengedit lingkungan dan menambahkan
rambu lalu lintas, sinyal lalu lintas, persimpangan, dan batas kendaraan. Selain
itu, RoadRunner dapat digunakan untuk membuat desain kota.

\section{Penelitian Terkait}
Subbab ini membahas beberapa penelitian terkait dengan Tugas Akhir ini.
Penelitian-penelitian berikut menjadi rujukan dalam penelitian dan pengembangan
Tugas Akhir ini.

% \subsection{Pengembangan Sistem Otonomi dengan Menggunakan Kecerdasan Artificial untuk Trem}
% % TODO: referensi papernya
% % yang "published" baru persepsinya?

% Penelitian ini mengembangkan sistem otonomi untuk Trem menggunakan kecerdasan
% buatan. Tugas Akhir ini merupakan bagian dari penelitian pada sisi simulasi
% untuk pengujian dan validasi strategi kemudi.

% \dots

\subsection{\textit{KIT Bus: A Shuttle Model for CARLA Simulator}}
Penelitian \textit{KIT Bus: A Shuttle Model for CARLA Simulator} ini merupakan
penelitian membuat model \textit{shuttle bus} untuk CARLA yang dilakukan oleh
\cite{related-work-xiang}. Proses pembuatan model tersebut dilakukan dalam tiga
tahap, yaitu sebagai berikut:
\begin{enumerate}
    \item Membuat model 3D dari bus menggunakan aplikasi 3ds Max.

    Model bus dibuat sesuai dengan referensi dan dimensi asli. Model bus yang
    dibuat secara detil dan berpermukaan mulus sehingga realistis. Model bus
    terdiri atas 6 bagian yaitu, bagian badan, bagian roda-roda, bagian
    interior, bagian detil, bagian kaca, dan bagian plat kendaraan.

    \item Melakukan pengeditan model 3D bus dalam CARLAUE4.

    Model bus yang telah dibuat kemudian diimpor ke dalam CARLAUE4.
    Masing-masing bagian badan bus dan keempat roda bus dipasangkan sebuah
    \textit{collision box} yang berbentuk dan berukuran sama. Agar model bus
    dapat menyimulasikan animasi bus yang sebenarnya, cetak biru animasi harus
    dibuat berbagai macam bagian bus yang bergerak. Dilakukan juga penambahan
    material atau tekstur dan fungsionalitas lainnya yang dibutuhkan seperti,
    lampu dan tekstur-tekstur yang ada pada permukaan bus. Hal tersebut
    dilakukan sehingga model bus dapat menyerupai model bus yang sebenarnya.
    Model bus yang telah selesai diedit dimasukkan ke dalam \textit{vehicle
    factory} agar dapat bisa dimunculkan (\textit{spawn}).

    \item Memverifikasi kontrol manual dan otonom dari model 3D bus yang telah
    dibangun dalam CARLA.

    Model bus yang telah dibuat kemudian diuji untuk memverifikasi apakah model
    bus tersebut dapat dioperasikan secara manual dan otonom. Model bus dites
    untuk berjalan maju, mengerem untuk melambat, mengemudi ke kiri dan ke
    kanan, dan memindahkan gigi kopling.

\end{enumerate}

\subsection{\textit{The Autonomous Siemens Tram}}
Penelitian ini membahas trem otonom Siemens yang telah didemonstrasikan di
Potsdam, Jerman pada tahun 2018. Sistem trem otonom tersebut dibangun di atas
trem Siemens Combino dan menggunakan sensor-sensor multi-modal untuk
mengidentifikasi lokasi kendaraan serta mendeteksi dan merespon lampu lalu
lintas dan objek lain. Trem otonom memiliki beberapa komputer yang memiliki
\textit{Graphic Processing Unit} (GPU) yang memadai untuk memroses data dari
sensor Lidar, sensor radar, kamera pendeteksi objek, dan kamera pendekteksi
sinyal. Trem otonom tersebut dapat beroperasi dengan lancar namun memiliki
kendala ketika melakukan lokalisasi trem hanya dengan Global Navigation
Satellite System (GNSS). Penelitian masih dilakukan untuk mengatasi kendala
tersebut, misalnya dengan melakukan lokalisasi dengan persepsi atau visual
\parencite{at-palmer}.
