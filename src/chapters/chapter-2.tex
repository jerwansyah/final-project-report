\chapter{Studi Literatur}

% Bab Studi Literatur digunakan untuk mendeskripsikan kajian literatur yang
% terkait dengan persoalan tugas akhir. Tujuan studi literatur adalah:

% \begin{enumerate}
%     \item menunjukkan kepada pembaca adanya gap seperti pada rumusan masalah
%         yang memang belum terselesaikan,
%     \item memberikan pemahaman yang secukupnya kepada pembaca tentang teori
%         atau pekerjaan terkait yang terkait langsung dengan penyelesaian
%         persoalan, serta
%     \item menyampaikan informasi apa saja yang sudah ditulis/dilaporkan oleh
%         pihak lain (peneliti/Tugas Akhir/Tesis) tentang hasil
%         penelitian/pekerjaan mereka yang sama atau mirip kaitannya dengan
%         persoalan tugas akhir.
% \end{enumerate}
Pada bab ini, Penulis akan menguraikan hasil literatur dalam penyusunan tugas
akhir ini. Subbab pertama akan membahas \dots .

Perujukan literatur dapat dilakukan dengan menambahkan entri baru di berkas.
Tulisan ini merujuk pada \parencite{knuth2001art,vasp1} atau
\parencite{4026885} dan \parencite{Kim2006} mungkin atau jg \parencite{dov17carla}

% Sekarang mau ke bab berapa yaaaa.... hmm... ke bab \ref{sec:latarbelakang} ahhhhh. 

\section{CARLA}
\blindtext

\section{Jalur Komunikasi pada Sistem Terdistribusi}
\blindtext

\section{Penelitian Terkait}
\blindtext

