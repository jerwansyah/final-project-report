\chapter{Pendahuluan}

% Bab Pendahuluan secara umum yang dijadikan landasan kerja dan arah kerja
% penulis Tugas Akhir, berfungsi mengantar pembaca untuk membaca laporan tugas
% akhir secara keseluruhan.

% note: preamble could be added

\section{Latar Belakang}

Perkembangan teknologi yang pesat memungkinkan terjadinya perkembangan teknologi
pada industri otomotif. Salah satu inovasi teknologi otomotif adalah kendaraan
yang dapat berjalan sendiri secara otonom tanpa adanya pengemudi manusia.
Kendaraan otonom tersebut dapat berkemudi sendiri dengan adanya kecerdasan
buatan. Teknologi kendaraan otonom ini tidak terbatas pada kendaraan pribadi
saja, tetapi dapat diaplikasikan juga pada kendaraan umum, seperti trem.

Pengembangan kendaraan otonom untuk penggunaan secara komersial perlu dilakukan
dengan teliti sehingga kendaraan otonom tersebut dapat beroperasi dengan baik,
dapat diandalkan, dapat nyaman digunakan, dapat meminimalisasi biaya bahan
bakar, dan dapat meminimalisasi kecelakaan lalu lintas. Kendaraan otonom yang
diharapkan tersebut membutuhkan pengembangan kecerdasan buatan yang dapat
mengoperasikan kendaraan yang spesifik. Oleh karena itu, pengembangan kendaraan
otonom membutuhkan biaya yang besar dan waktu yang lama sebab dibutuhkan data
yang banyak dan bervariasi untuk melatih kecerdasan buatan. Selain itu,
dibutuhkan juga pengujian kendaraan otonom untuk melakukan validasi kecerdasan
buatan yang telah dibuat. Pengumpulan data untuk pelatihan dapat membutuhkan
waktu yang lama sebab terdapat banyak faktor eksternal yang tidak dapat
diprediksi atau dikontrol yang membuat data yang dikumpulkan kurang bervariasi.
Proses untuk mencari data yang bervariasi dapat memakan waktu lama sebab harus
menunggu saat yang tepat untuk pengumpulan data. Selain pengumpulan data,
pengujian kendaraan otonom juga membutuhkan waktu yang lama dan biaya yang besar
untuk memastikan kendaraan otonom dapat beroperasi dengan baik, pengujian harus
dilakukan berulang-ulang dan dalam berbagai skenario yang kadang tidak dapat
diatur sesuai dengan keinginan di saat yang diinginkan.

Pelatihan dan pengujian kendaraan otonom yang mahal tersebut dapat diatasi
dengan melakukannya secara virtual, yaitu menggunakan simulasi. Pelatihan dan
pengujian dengan menggunakan simulasi dapat menghemat biaya dan waktu karena
aktor, lingkungan, dan parameter simulasi untuk pengujian dapat diatur sesuai
dengan keinginan relatif lebih mudah dan lebih cepat.

Lingkungan atau dunia simulasi yang identik dengan dunia nyata diperlukan
sehingga kecerdasan buatan kendaraan otonom cocok digunakan untuk dunia pada
kenyataan. Oleh karena itu, dibutuhkan integrasi atau implementasi aktor/objek
dinamis, dan dunia simulasi. Dunia simulasi saat ini belum serupa dengan
lingkungan Indonesia.

Tugas Akhir ini membahas mengenai implementasi objek trem, implementasi objek
lalu lintas khas Indonesia, dan implementasi dunia simulasi yang serupa dengan
Indonesia menggunakan simulator Car Learning to Act (CARLA) untuk pengembangan
trem otonom. Aset bawaan dari CARLA seperti kendaraan, jalan, kota, dunia atau
lingkungan, dan aset lainnya mencerminkan objek dan lingkungan di Amerika
Serikat pada umumnya. Oleh karena itu, perlu dibuat aset yang mencerminkan objek
dan lingkungan di Indonesia.

% Lalu lintas \textit{default} di CARLA adalah jalur kanan. Hal tersebut juga
% perlu disesuaikan agar sama seperti lalu lintas di Indonesia.

\section{Rumusan Masalah}

Berdasarkan latar belakang yang telah dibahas, rumusan masalah yang akan dibahas
dalam Tugas Akhir ini adalah sebagai berikut.

\begin{enumerate}

    \item Bagaimana cara mengimplementasikan trem dan objek lokal khas
    Indonesia, yaitu angkot, sepeda onthel, sepeda motor, dan becak
    di dunia simulasi simulator CARLA?

    \item Bagaimana cara mengimplementasikan dunia simulasi yang serupa
    dengan lingkungan Indonesia di simulator CARLA?

\end{enumerate}

\section{Tujuan}

Tujuan dari Tugas Akhir ini adalah sebagai berikut:

\begin{enumerate}

    \item Mengimplementasikan trem dan objek lokal khas Indonesia yang stabil di
    dunia simulasi simulator CARLA.

    \item Mengimplementasikan dunia simulasi yang serupa dengan cara
    mengimplementasikan stasiun, rambu lalu lintas, dan rel di dunia simulasi
    simulator CARLA.

    % \item Membuat distribusi hasil implementasi.

\end{enumerate}

\section{Batasan Masalah}

Batasan-batasan masalah Tugas Akhir ini adalah sebagai berikut:

\begin{enumerate}

    \item Implementasi objek trem, objek lokal khas Indonesia, dan dunia
    simulasi hanya terbatas pada lingkungan Kota Solo.

    \item Tingkat kemiripan simulasi dan hasil implementasi terbatas pada
    aplikasi simulator CARLA versi 0.9.12 dan/atau versi 0.9.13.

    % \item Distribusi hasil implementasi dibuat hanya untuk sistem operasi Linux
    % (Ubuntu).

\end{enumerate}

\section{Metodologi}

Berikut adalah metodologi yang digunakan dalam pengerjaan Tugas Akhir ini:

\begin{enumerate}

    \item Analisis masalah

    Melakukan analisis terhadap rumusan masalah mengenai simulasi trem otonom.
    Analisis dilakukan dengan melakukan studi literatur mengenai simulasi
    kendaraan otonom.

    \item Eksplorasi kakas dan dokumentasi kakas

    Melakukan eksplorasi kakas yang akan digunakan serta dokumentasi kakas untuk
    mengetahui fitur-fitur kakas yang dapat digunakan untuk implementasi.

    \item Implementasi

    Melakukan implementasi objek trem, objek lokal khas Indonesia, dan
    dunia simulasi di CARLA Unreal Engine 4 atau CARLAUE4 (editor simulator
    CARLA) agar dapat digunakan di simulator CARLA.

    \item Validasi dan evaluasi implementasi

    Pada tahap ini dilakukan validasi dan evaluasi hasil implementasi. Tahap ini
    dilakukan untuk memastikan bahwa objek trem, objek lokal khas Indonesia, dan
    dunia simulasi yang diimplementasikan berperilaku normal, stabil untuk
    digunakan dalam simulasi, dan dapat memenuhi kebutuhan simulasi.

\end{enumerate}

\section{Sistematika Pembahasan}

Laporan Tugas Akhir ini terdiri atas 5 bab, yaitu Pendahuluan, Studi Literatur,
Analisis Masalah dan Rancangan Implementasi, Implementasi dan Evaluasi, dan
Kesimpulan dan Saran.

Bab I Pendahuluan menjelaskan latar belakang masalah, rumusan masalah
berdasarkan latar belakang, tujuan, batasan masalah dari implementasi tujuan,
metodologi pengerjaan, dan sistematika pembahasan laporan.

Bab II Studi literatur menjelaskan mengenai hasil studi literatur yang
dibutuhkan untuk menganalisis masalah dan mengimplementasikan rancangan untuk
menyelesaikan masalah serta beberapa penelitian yang terkait mengenai Tugas
Akhir ini.

Bab III Analisis Masalah dan Rancangan Implementasi menjelaskan mengenai
deskripsi umum permasalahan \textit{capstone}, analisis masalah, analisis
solusi, dan rancangan implementasi.

Bab IV Implementasi dan Evaluasi menjelaskan mengenai hasil eksplorasi kakas,
hasil implementasi, validasi hasil implementasi, evaluasi
implementasi, dan distribusi hasil implementasi.

Bab V Kesimpulan dan Saran menjelaskan mengenai kesimpulan dari pengerjaan Tugas
Akhir dan saran untuk pengembangan lanjut Tugas Akhir.
