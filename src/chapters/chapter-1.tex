\chapter{Pendahuluan}

% Bab Pendahuluan secara umum yang dijadikan landasan kerja dan arah kerja
% penulis tugas akhir, berfungsi mengantar pembaca untuk membaca laporan tugas
% akhir secara keseluruhan.

\section{Latar Belakang}

Perkembangan teknologi yang pesat memungkinkan terjadinya perkembangan teknologi
pada industri otomotif. Salah satu inovasi teknologi otomotif adalah kendaraan
yang dapat berjalan sendiri secara otonom tanpa adanya pengemudi manusia.
Kendaraan otonom tersebut dapat berkemudi sendiri dengan adanya kecerdasan
buatan. Teknologi kendaraan otonom ini tidak terbatas pada kendaraan pribadi
saja, tetapi dapat diaplikasikan juga pada kendaraan umum seperti trem.

Pengembangan kendaraan otonom untuk penggunaan secara komersial perlu dilakukan
dengan teliti sehingga kendaraan otonom tersebut dapat beroperasi dengan baik,
dapat diandalkan, dapat nyaman digunakan, dapat meminimalisasikan biaya bahan
bakar karena menggunakan listrik, dan dapat memimalisasi kecelakaan lalu lintas.
Kendaraan otonom yang diharapkan tersebut membutuhkan pengembangan kecerdasan
buatan yang dapat mengoperasikan kendaraan yang spesifik. Oleh karena itu,
pengembangan kendaraan otonom membutuhkan biaya yang besar dan waktu yang lama
sebab dibutuhkan data yang banyak dan bervariasi untuk melatih kecerdasan
buatan. Selain itu, dibutuhkan juga pengujian kendaraan otonom untuk melakukan
validasi kecerdasan buatan yang telah dibuat. Pengumpulan data untuk pelatihhan
dapat membutuhkan waktu yang lama sebab terdapat banyak faktor eksternal yang
tidak dapat diprediksi atau dikontrol yang membuat data yang dikumpulkan tidak
bervariasi. Proses untuk mencari data yang bervariasi dapat memakan waktu juga
sebab harus menunggu saat yang tepat untuk pengumpulan data. Selain pengumpulan
data, pengujian kendaraan otonom juga membutuhkan waktu yang lama dan biaya yang
besar sebab untuk memastikan kendaraan otonom dapat beroperasi dengan baik,
pengujian harus dilakukan berulang-ulang dan dalam berbagai skenario yang kadang
tidak dapat diatur sesuai dengan keinginan di saat yang diinginkan.

Pelatihan dan pengujian kendaraan otonom yang mahal ini dapat diatasi dengan
melakukannya secara virtual menggunakan aplikasi simulasi. Pelatihan dan
pengujian dengan menggunakan simulasi dapat menghemat biaya dan waktu karena
aktor, lingkungan, dan parameter simulasi untuk pengujian dapat diatur sesuai
dengan keinginan relatif lebih mudah dan lebih cepat. Lingkungan simulasi yang
identik dengan lingkungan nyata diperlukan sehingga kecerdasan buatan kendaraan
otonom cocok digunakan untuk lingkungan pada kenyataan. Oleh karena itu,
dibutuhkan implementasi aktor, objek, dan lingkungan simulasi.

Tugas akhir ini membahas implementasi objek trem, implementasi objek lalu lintas
khas Indonesia, dan implementasi lingkungan simulasi yang serupa dengan
Indonesia menggunakan simulator Car Learning to Act (CARLA). Aset bawaan dari
CARLA seperti kendaraan, jalan, kota, lingkungan, dan aset lainnya mencerminkan
objek dan lingkungan di Amerika Serikat pada umumnya. Oleh karena itu, perlu
dibuat aset yang mencerminkan objek dan lingkungan di Indonesia. Lalu lintas
\textit{default} di CARLA adalah jalur kanan. Hal tersebut juga perlu
disesuaikan agar sama seperti lalu lintas di Indonesia.

\section{Rumusan Masalah}

Berdasarkan latar belakang yang telah dibahas, rumusan masalah yang akan dibahas
dalam tugas akhir ini adalah sebagai berikut.
\begin{enumerate}
	\item Bagaimana cara mengimplementasikan objek trem, objek lokal khas
	Indonesia di lingkungan simulator CARLA?
	\item Bagaimana cara mengimplementasikan lingkungan simulasi yang serupa
	dengan lingkungan Indonesia di simulator CARLA?
\end{enumerate}

\section{Tujuan}

Tujuan dari tugas akhir ini adalah untuk mengimplentasikan objek trem, objek
lokal khas Indonesia, dan lingkungan khas Indonesia di simulator CARLA untuk
memudahkan validasi algortima \textit{decision making} trem otonom.

\section{Batasan Masalah}

Penelititan untuk tugas akhir ini memiliki batasan masalah yaitu
pengimplementasian objek trem, objek lokal khas Indonesia, dan lingkungan
simulasi hanya terbatas pada lingkungan kota Solo. Kota Solo dipilih sebagai
lingkungan pengujian trem otonom karena di Solo sudah terdapat trem konvensional
yang beroperasi. Pengambilan data untuk pelatihan telah dan sedang dilakukan di
Solo. Pengujian trem otonom di luar simulasi pun akan dilakukan di Solo.
Tingkat kemiripan simulasi terbatas pada aplikasi simulator CARLA.

\section{Metodologi}

Metodologi yang digunakan dalam pengerjaan tugas akhir ini adalah sebagai
berikut.

\begin{enumerate}
	\item Analisis masalah

	Melakukan analisis terhadap rumusan masalah mengenai simulasi trem otonom.
	Analisis dilakukan dengan melakukan studi literatur mengenai simulasi
	kendaraan otonom.

	\item Eksplorasi kakas untuk mengimplementasikan solusi

	Melakukan eksplorasi kakas yang akan digunakan untuk implementasi dan
	validasi untuk menyelesaikan masalah yang telah dirumuskan.

	\item Implementasi solusi

	Melakukan implementasi objek trem, objek lokal khas Indonesia, dan
	lingkungan simulasi di \textit{editor} simulator CARLA (CARLA Unreal Engine
	4 Editor) agar dapat digunakan di simulator CARLA.

	\item Validasi dan analisis implementasi solusi

	Pada tahap ini dilakukan validasi dan analisis solusi yang sudah
	implementasi. Tahap ini dilakukan untuk memastikan bahwa objek trem, objek
	lokal khas Indonesia, dan lingkungan simulasi yang diimplementasikan
	berperilaku normal, stabil untuk digunakan dalam simulasi, dan dapat
	memenuhi kebutuhan simulasi.

\end{enumerate}

\section{Jadwal Pelaksanaan Tugas Akhir}

Berikut adalah jadwal pelaksanaan Tugas Akhir I perminggunya.

\begin{figure}[ht]
	\linespread{0.8}
	\resizebox{\textwidth}{!}{
		\begin{ganttchart}[
				y unit title=0.5cm,
				y unit chart=1.3cm,
				bar height=0.7,
				vgrid,
				title height=1,
				bar label font=\tiny,
				bar label node/.style={
						text width=4cm,
						align=right,
						anchor=east,
						font=\raggedleft
					},
			]{1}{20}
			%labels
			\gantttitle{September}{4}
			\gantttitle{Oktober}{4}
			\gantttitle{November}{4}
			\gantttitle{Desember}{4}
			\gantttitle{Januari}{4} \\
			\gantttitlelist{1,...,4}{1}
			\gantttitlelist{1,...,4}{1}
			\gantttitlelist{1,...,4}{1}
			\gantttitlelist{1,...,4}{1}
			\gantttitlelist{1,...,4}{1} \\

			% tasks
			\ganttbar{Mempelajari Dokumentasi CARLA}{4}{4} \\
			\ganttbar{Eksplorasi CARLA}{5}{7} \\
			\ganttbar{Implementasi Objek Angkot}{8}{10} \\
			\ganttbar{Implementasi Objek Tram}{11}{13} \\
			\ganttbar{Implementasi Objek Kendaraan}{14}{16} \\
            \ganttbar{Implementasi Lingkungan Baru}{17}{20} \\
            \ganttbar{Sidang Tugas Akhir 1}{17}{18} \\
		\end{ganttchart}
	}
	% \caption{Jadwal Pelaksanaan Tugas Akhir (bagian 1)}
	\caption{Jadwal Pelaksanaan Tugas Akhir}
\end{figure}

% TODO
% \begin{figure}[ht]
% 	\begin{center}
% 		\linespread{0.8}
% 		\resizebox{\textwidth}{!}{
% 			\begin{ganttchart}[
% 					y unit title=0.5cm,
% 					y unit chart=1.3cm,
% 					bar height=0.7,
% 					vgrid,
% 					title height=1,
% 					bar label font=\tiny,
% 					bar label node/.style={
% 							text width=4cm,
% 							align=right,
% 							anchor=east,
% 							font=\raggedleft
% 						},
% 				]{1}{24}
% 				%labels
% 				\gantttitle{Februari}{4}
% 				\gantttitle{Maret}{4}
% 				\gantttitle{April}{4}
% 				\gantttitle{Mei}{4}
% 				\gantttitle{Juni}{4}
% 				\gantttitle{Juli}{4} \\
% 				\gantttitlelist{1,...,4}{1}
% 				\gantttitlelist{1,...,4}{1}
% 				\gantttitlelist{1,...,4}{1}
% 				\gantttitlelist{1,...,4}{1}
% 				\gantttitlelist{1,...,4}{1}
% 				\gantttitlelist{1,...,4}{1} \\

% 				% tasks
% 				\ganttbar{Perancangan kerangka kerja pengujian}{1}{8} \\
% 				\ganttbar{Implementasi kerangka kerja pengujian}{9}{16} \\
% 			\end{ganttchart}
% 		}
% 	\end{center}
% 	\caption{Jadwal Pelaksanaan Tugas Akhir (bagian 2)}
% \end{figure}