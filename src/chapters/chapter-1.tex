\chapter{Pendahuluan}

% Bab Pendahuluan secara umum yang dijadikan landasan kerja dan arah kerja
% penulis Tugas Akhir, berfungsi mengantar pembaca untuk membaca laporan tugas
% akhir secara keseluruhan.

\section{Latar Belakang}

Perkembangan teknologi yang pesat memungkinkan terjadinya perkembangan teknologi
pada industri otomotif. Salah satu inovasi teknologi otomotif adalah kendaraan
yang dapat berjalan sendiri secara otonom tanpa adanya pengemudi manusia.
Kendaraan otonom tersebut dapat berkemudi sendiri dengan adanya kecerdasan
buatan. Teknologi kendaraan otonom ini tidak terbatas pada kendaraan pribadi
saja, tetapi dapat diaplikasikan juga pada kendaraan umum, seperti trem.

Pengembangan kendaraan otonom untuk penggunaan secara komersial perlu dilakukan
dengan teliti sehingga kendaraan otonom tersebut dapat beroperasi dengan baik,
dapat diandalkan, dapat nyaman digunakan, dapat meminimalisasikan biaya bahan
bakar karena menggunakan listrik, dan dapat memimalisasi kecelakaan lalu lintas.
Kendaraan otonom yang diharapkan tersebut membutuhkan pengembangan kecerdasan
buatan yang dapat mengoperasikan kendaraan yang spesifik. Oleh karena itu,
pengembangan kendaraan otonom membutuhkan biaya yang besar dan waktu yang lama
sebab dibutuhkan data yang banyak dan bervariasi untuk melatih kecerdasan
buatan. Selain itu, dibutuhkan juga pengujian kendaraan otonom untuk melakukan
validasi kecerdasan buatan yang telah dibuat. Pengumpulan data untuk pelatihan
dapat membutuhkan waktu yang lama sebab terdapat banyak faktor eksternal yang
tidak dapat diprediksi atau dikontrol yang membuat data yang dikumpulkan tidak
bervariasi. Proses untuk mencari data yang bervariasi dapat memakan waktu juga
sebab harus menunggu saat yang tepat untuk pengumpulan data. Selain pengumpulan
data, pengujian kendaraan otonom juga membutuhkan waktu yang lama dan biaya yang
besar sebab untuk memastikan kendaraan otonom dapat beroperasi dengan baik,
pengujian harus dilakukan berulang-ulang dan dalam berbagai skenario yang kadang
tidak dapat diatur sesuai dengan keinginan di saat yang diinginkan.

Pelatihan dan pengujian kendaraan otonom yang mahal tersebut dapat diatasi
dengan melakukannya secara virtual menggunakan aplikasi simulasi. Pelatihan dan
pengujian dengan menggunakan simulasi dapat menghemat biaya dan waktu karena
aktor, lingkungan, dan parameter simulasi untuk pengujian dapat diatur sesuai
dengan keinginan relatif lebih mudah dan lebih cepat.

Penelitian "Pengembangan Sistem Otonomi dengan Menggunakan Kecerdasan
\textit{Artificial} untuk Trem" (selanjutnya akan dirujuk sebagai proyek trem
otonom) merupakan penelitian mengenai pengembangan trem otonom (lihat Subbab
\ref{subsec:rispro-trilaksono}). Penelitian ini menggunakan simulasi untuk
mengembangkan pembelajaran mesin (\textit{Machine Learning}) dan/atau kecerdasan
buatan (\textit{Artificial Intelligence}), menguji operasi, dan memvalidasi
operasi trem otonom pada tahap-tahap awal atau sebelum diuji dan divalidasi di
lingkungan nyata dengan tujuan mendukung pengembangan trem otonom agar lebih
cepat, lebih mudah dan lebih hemat biaya. Terdapat empat tim yang bekerja sama
untuk mengembangkan kecerdasan buatan trem otonom, yaitu:

\begin{enumerate}

    \item Tim persepsi

    Tim persepsi merupakan tim yang bertanggung jawab untuk mengembangkan model
    persepsi yang berfungsi untuk mengolah data sensor-sensor (kamera, radar,
    LIDAR).

    \item Tim \textit{localization} dan \textit{mapping}

    Tim \textit{localization} dan \textit{mapping} merupakan tim yang
    bertanggung jawab untuk memetakan data lokasi tram dari sensor GPS.

    \item Tim \textit{decision making}

    Tim \textit{decision making} merupakan tim yang bertanggung jawab untuk
    mengembangkan modul pengambilan keputusan. Modul tersebut berfungsi untuk
    melakukan \textit{risk assessment} dari hasil persepsi,
    \textit{localization}, dan \textit{mapping} kemudian menentukan keputusan
    dan \textit{planning}. Hasil dari modul tersebut adalah perintah aktuator
    untuk mengendalikan/mengemudikan trem.

    \item Tim simulasi

    Tim simulasi merupakan tim yang bertanggung jawab untuk mengembangkan
    \textit{Software-in-the-Loop Simulation} (SILS) dan
    \textit{Hardware-in-the-Loop Simulation} (HILS). Simulasi berskema HILS
    dilakukan menggunakan perangkat keras NVIDIA DRIVE AGX Pegasus untuk
    memroses \textit{decision making}, memroses persepsi, memroses
    \textit{localization}, dan memroses \textit{mapping}. Simulasi menggunakan
    simulator CARLA yang berada di komputer atau perangkat keras yang berbeda.
    Simulator ini digunakan untuk menyimulasikan trem dan lingkungan virtual
    untuk mendapatkan data sensor virtual. Tim \textit{capstone} Tugas Akhir ini
    merupakan bagian dari tim simulasi.

\end{enumerate}

Lingkungan simulasi yang identik dengan lingkungan nyata diperlukan sehingga
kecerdasan buatan kendaraan otonom cocok digunakan untuk lingkungan pada
kenyataan. Oleh karena itu, dibutuhkan integrasi atau implementasi aktor, objek,
dan lingkungan simulasi. Lingkungan simulasi saat ini belum serupa dengan
lingkungan Indonesia.

Tugas Akhir ini membahas mengenai implementasi objek trem, implementasi objek
lalu lintas khas Indonesia, dan implementasi lingkungan simulasi yang serupa
dengan Indonesia menggunakan simulator Car Learning to Act (CARLA) untuk
pengembangan trem otonom pada proyek trem otonom. Aset bawaan dari CARLA seperti
kendaraan, jalan, kota, lingkungan, dan aset lainnya mencerminkan objek dan
lingkungan di Amerika Serikat pada umumnya. Oleh karena itu, perlu dibuat aset
yang mencerminkan objek dan lingkungan di Indonesia. Lalu lintas
\textit{default} di CARLA adalah jalur kanan. Hal tersebut juga perlu
disesuaikan agar sama seperti lalu lintas di Indonesia.

\section{Rumusan Masalah}

Berdasarkan latar belakang yang telah dibahas, rumusan masalah yang akan dibahas
dalam Tugas Akhir ini adalah sebagai berikut.

\begin{enumerate}

    \item Bagaimana cara mengimplementasikan objek trem, objek lokal khas
    Indonesia di lingkungan simulator CARLA?

    \item Bagaimana cara mengimplementasikan lingkungan simulasi yang serupa
    dengan lingkungan Indonesia di simulator CARLA?

\end{enumerate}

\section{Tujuan}

Tujuan dari Tugas Akhir ini adalah untuk mengimplentasikan objek trem, objek
lokal khas Indonesia, dan lingkungan khas Indonesia di simulator CARLA untuk
memudahkan pengujian dan validasi algortima \textit{decision making} trem
otonom.

\section{Batasan Masalah}

Penelititan untuk Tugas Akhir ini memiliki batasan masalah, yaitu implementasi
objek trem, objek lokal khas Indonesia, dan lingkungan simulasi hanya terbatas
pada lingkungan kota Solo. Kota Solo dipilih sebagai lingkungan pengujian trem
otonom karena di Solo sudah terdapat trem konvensional yang beroperasi.
Pengambilan data untuk pelatihan telah dan sedang dilakukan di Solo. Pengujian
trem otonom di luar simulasi pun akan dilakukan di Solo. Tingkat kemiripan
simulasi terbatas pada aplikasi simulator CARLA versi 0.9.12 dan/atau versi
0.9.13.

\section{Metodologi}

Metodologi yang digunakan dalam pengerjaan Tugas Akhir ini adalah sebagai
berikut:

\begin{enumerate}

    \item Analisis masalah

    Melakukan analisis terhadap rumusan masalah mengenai simulasi trem otonom.
    Analisis dilakukan dengan melakukan studi literatur mengenai simulasi
    kendaraan otonom.

    \item Eksplorasi kakas untuk mengimplementasikan solusi

    Melakukan eksplorasi kakas yang akan digunakan untuk implementasi dan
    validasi untuk menyelesaikan masalah yang telah dirumuskan.

    \item Implementasi solusi

    Melakukan implementasi objek trem, objek lokal khas Indonesia, dan
    lingkungan simulasi di \textit{editor} simulator CARLA (CARLA Unreal Engine
    4 Editor) agar dapat digunakan di simulator CARLA.

    \item Validasi dan analisis implementasi solusi

    Pada tahap ini dilakukan validasi dan analisis solusi yang sudah
    diimplementasi. Tahap ini dilakukan untuk memastikan bahwa objek trem, objek
    lokal khas Indonesia, dan lingkungan simulasi yang diimplementasikan
    berperilaku normal, stabil untuk digunakan dalam simulasi, dan dapat
    memenuhi kebutuhan simulasi.

\end{enumerate}

\section{Sistematika Pembahasan}

Sistematika pembahasan laporan Tugas Akhir ini adalah sebagai berikut:

\begin{enumerate}

    \item Bab I Pendahuluan menjelaskan latar belakang masalah, rumusan masalah
    berdasarkan latar belakang, tujuan Tugas Akhir, batasan masalah dari
    implementasi tujuan, metodologi pengerjaan Tugas Akhir, dan sistematika
    pembahasan laporan Tugas Akhir.

    \item Bab II Studi literatur menjelaskan mengenai hasil studi yang
    dibutuhkan untuk menganalisis masalah dan mengimplementasikan solusi untuk
    menyelesaikan masalah. Penelitian terkait mengenai Tugas Akhir ini.

    \item Bab III Analisis dan Rancangan Implementasi Objek dan Lingkungan
    menjelaskan analisis masalah, analisis solusi, dan rancangan implementasi
    solusi.

    \item Bab IV Implementasi Objek dan Lingkungan menjelaskan implementasi
    solusi yang dibuat.

    \item Bab V Kesimpulan dan Saran menjelaskan mengenai kesimpulan dan saran
    pengerjaan tugas akhir dan penyusunan laporan tugas akhir.

\end{enumerate}
