\chapter{Pendahuluan}

% Bab Pendahuluan secara umum yang dijadikan landasan kerja dan arah kerja
% penulis Tugas Akhir, berfungsi mengantar pembaca untuk membaca laporan tugas
% akhir secara keseluruhan.

% note: preamble could be added

\section{Latar Belakang}

Perkembangan teknologi yang pesat memungkinkan terjadinya perkembangan teknologi
pada industri otomotif. Salah satu inovasi teknologi otomotif adalah kendaraan
yang dapat berjalan sendiri secara otonom tanpa adanya pengemudi manusia.
Kendaraan otonom tersebut dapat berkemudi sendiri dengan adanya kecerdasan
buatan. Teknologi kendaraan otonom ini tidak terbatas pada kendaraan pribadi
saja, tetapi dapat diaplikasikan juga pada kendaraan umum, seperti trem.

Pengembangan kendaraan otonom untuk penggunaan secara komersial perlu dilakukan
dengan teliti sehingga kendaraan otonom tersebut dapat beroperasi dengan baik,
dapat diandalkan, dapat nyaman digunakan, dapat meminimalisasi biaya bahan bakar
karena menggunakan listrik, dan dapat meminimalisasi kecelakaan lalu lintas.
Kendaraan otonom yang diharapkan tersebut membutuhkan pengembangan kecerdasan
buatan yang dapat mengoperasikan kendaraan yang spesifik. Oleh karena itu,
pengembangan kendaraan otonom membutuhkan biaya yang besar dan waktu yang lama
sebab dibutuhkan data yang banyak dan bervariasi untuk melatih kecerdasan
buatan. Selain itu, dibutuhkan juga pengujian kendaraan otonom untuk melakukan
validasi kecerdasan buatan yang telah dibuat. Pengumpulan data untuk pelatihan
dapat membutuhkan waktu yang lama sebab terdapat banyak faktor eksternal yang
tidak dapat diprediksi atau dikontrol yang membuat data yang dikumpulkan tidak
bervariasi. Proses untuk mencari data yang bervariasi dapat memakan waktu juga
sebab harus menunggu saat yang tepat untuk pengumpulan data. Selain pengumpulan
data, pengujian kendaraan otonom juga membutuhkan waktu yang lama dan biaya yang
besar sebab untuk memastikan kendaraan otonom dapat beroperasi dengan baik,
pengujian harus dilakukan berulang-ulang dan dalam berbagai skenario yang kadang
tidak dapat diatur sesuai dengan keinginan di saat yang diinginkan.

Pelatihan dan pengujian kendaraan otonom yang mahal tersebut dapat diatasi
dengan melakukannya secara virtual menggunakan aplikasi simulasi. Pelatihan dan
pengujian dengan menggunakan simulasi dapat menghemat biaya dan waktu karena
aktor, lingkungan, dan parameter simulasi untuk pengujian dapat diatur sesuai
dengan keinginan relatif lebih mudah dan lebih cepat.

Tim \textit{capstone} Tugas Akhir ini merupakan bagian dari tim penelitian
``Pengembangan Sistem Otonomi dengan Menggunakan Kecerdasan \textit{Artificial}
untuk Trem" (selanjutnya akan disebut sebagai proyek trem otonom). Penelitian
tersebut merupakan penelitian mengenai pengembangan trem otonom. Penelitian
tersebut menggunakan simulasi untuk mengembangkan pembelajaran mesin
(\textit{Machine Learning}) dan/atau kecerdasan buatan (\textit{Artificial
Intelligence}), menguji operasi, dan memvalidasi operasi trem otonom pada
tahap-tahap awal atau sebelum diuji dan divalidasi di lingkungan nyata. Tujuan
menggunakan simulasi adalah untuk mendukung pengembangan trem otonom agar lebih
cepat, lebih mudah dan lebih hemat biaya. Pengambilan data untuk pelatihan dan
pengujian algoritma-algoritma trem otonom diperoleh dari lingkungan di Kota
Solo. Pengujian trem otonom di luar simulasi juga akan dilakukan di Kota Solo.
Pengambilan data dan pengujian dilakukan di Kota Solo karena di Kota Solo sudah
terdapat trem konvensional yang beroperasi. Subbab
\ref{subsec:rispro-trilaksono} menjelaskan lebih lanjut mengenai penelitian
tersebut.

Terdapat 4 tim yang bekerja sama untuk mengembangkan kecerdasan buatan trem
otonom, yaitu:

\begin{enumerate}

    \item Tim persepsi

    Tim persepsi merupakan tim yang bertanggung jawab untuk mengembangkan model
    persepsi yang berfungsi untuk mengolah data sensor-sensor (kamera, radar,
    LIDAR).

    \item Tim \textit{localization} dan \textit{mapping}

    Tim \textit{localization} dan \textit{mapping} merupakan tim yang
    bertanggung jawab untuk memetakan data lokasi tram dari sensor GPS.

    \item Tim \textit{decision making}

    Tim \textit{decision making} merupakan tim yang bertanggung jawab untuk
    mengembangkan modul pengambilan keputusan. Modul tersebut berfungsi untuk
    melakukan \textit{risk assessment} dari hasil persepsi,
    \textit{localization}, dan \textit{mapping} kemudian menentukan keputusan
    dan \textit{planning}. Hasil dari modul tersebut adalah perintah aktuator
    untuk mengendalikan/mengemudikan trem.

    \item Tim simulasi

    Tim simulasi merupakan tim yang bertanggung jawab untuk mengembangkan
    \textit{Software-in-the-Loop Simulation} (SILS) dan
    \textit{Hardware-in-the-Loop Simulation} (HILS). Simulasi berskema HILS
    dilakukan menggunakan perangkat keras NVIDIA DRIVE AGX Pegasus untuk
    memroses \textit{decision making}, memroses persepsi, memroses
    \textit{localization}, dan memroses \textit{mapping}. Simulasi menggunakan
    simulator CARLA yang berada di komputer atau perangkat keras yang berbeda.
    Simulator ini digunakan untuk menyimulasikan trem dan lingkungan virtual
    untuk mendapatkan data sensor virtual. Tim \textit{capstone} Tugas Akhir ini
    merupakan bagian dari tim simulasi.

\end{enumerate}

Lingkungan simulasi yang identik dengan lingkungan nyata diperlukan sehingga
kecerdasan buatan kendaraan otonom cocok digunakan untuk lingkungan pada
kenyataan. Oleh karena itu, dibutuhkan integrasi atau implementasi aktor, objek,
dan lingkungan simulasi. Lingkungan simulasi saat ini belum serupa dengan
lingkungan Indonesia.

Tugas Akhir ini membahas mengenai implementasi objek trem, implementasi objek
lalu lintas khas Indonesia, dan implementasi lingkungan simulasi yang serupa
dengan Indonesia menggunakan simulator Car Learning to Act (CARLA) untuk
pengembangan trem otonom pada proyek trem otonom. Aset bawaan dari CARLA seperti
kendaraan, jalan, kota, lingkungan, dan aset lainnya mencerminkan objek dan
lingkungan di Amerika Serikat pada umumnya. Oleh karena itu, perlu dibuat aset
yang mencerminkan objek dan lingkungan di Indonesia. Lalu lintas
\textit{default} di CARLA adalah jalur kanan. Hal tersebut juga perlu
disesuaikan agar sama seperti lalu lintas di Indonesia.

\section{Rumusan Masalah}

Berdasarkan latar belakang yang telah dibahas, rumusan masalah yang akan dibahas
dalam Tugas Akhir ini adalah sebagai berikut.

\begin{enumerate}

    \item Bagaimana cara mengimplementasikan trem dan objek lokal khas
    Indonesia, yaitu angkot, sepeda onthel, sepeda motor, dan becak
    di lingkungan simulasi simulator CARLA?

    \item Bagaimana cara mengimplementasikan lingkungan simulasi yang serupa
    dengan lingkungan Indonesia di simulator CARLA?

    % \item Bagaimana cara mengimplementasikan trem dan objek lokal khas
    % Indonesia, yaitu angkot, sepeda onthel, sepeda motor, dan becak di
    % simulasi simulator CARLA?

    % \item Bagaimana cara mengimplementasikan stasiun, rambu lalu lintas, dan rel
    % di lingkungan simulasi simulator CARLA?

\end{enumerate}

\section{Tujuan}

Tujuan dari Tugas Akhir ini adalah sebagai berikut:

\begin{enumerate}

    \item Mengimplementasikan trem dan objek lokal khas Indonesia yang stabil di
    lingkungan simulasi simulator CARLA.

    \item Mengimplementasikan lingkungan simulasi yang serupa dengan cara
    mengimplementasikan stasiun, rambu lalu lintas, dan rel di lingkungan
    simulasi simulator CARLA.

    \item Membuat distribusi hasil implementasi.

\end{enumerate}

\section{Batasan Masalah}

Batasan-batasan masalah Tugas Akhir ini adalah sebagai berikut:

\begin{enumerate}

    \item Implementasi objek trem, objek lokal khas Indonesia, dan lingkungan
    simulasi hanya terbatas pada lingkungan Kota Solo.

    \item Tingkat kemiripan simulasi dan hasil implementasi terbatas pada
    aplikasi simulator CARLA versi 0.9.12 dan/atau versi 0.9.13.

    \item Distribusi hasil implementasi dibuat hanya untuk sistem operasi Linux
    (Ubuntu).

\end{enumerate}

\section{Metodologi}

Metodologi yang digunakan dalam pengerjaan Tugas Akhir ini adalah sebagai
berikut:

\begin{enumerate}

    \item Analisis masalah

    Melakukan analisis terhadap rumusan masalah mengenai simulasi trem otonom.
    Analisis dilakukan dengan melakukan studi literatur mengenai simulasi
    kendaraan otonom.

    \item Eksplorasi kakas dan dokumentasi kakas

    Melakukan eksplorasi kakas yang akan digunakan serta dokumentasi kakas untuk
    mengetahui fitur-fitur kakas yang dapat digunakan untuk implementasi solusi.

    \item Implementasi solusi

    Melakukan implementasi objek trem, objek lokal khas Indonesia, dan
    lingkungan simulasi di CARLA Unreal Engine 4 atau CARLAUE4 (editor simulator
    CARLA) agar dapat digunakan di simulator CARLA.

    \item Validasi dan analisis implementasi solusi

    Pada tahap ini dilakukan validasi dan analisis solusi yang sudah
    diimplementasi. Tahap ini dilakukan untuk memastikan bahwa objek trem, objek
    lokal khas Indonesia, dan lingkungan simulasi yang diimplementasikan
    berperilaku normal, stabil untuk digunakan dalam simulasi, dan dapat
    memenuhi kebutuhan simulasi.

\end{enumerate}

\section{Sistematika Pembahasan}

Laporan Tugas Akhir ini terdiri atas 5 bab, yaitu Pendahuluan, Studi Literatur,
Deskripsi Solusi, Implementasi dan Evaluasi, dan Kesimpulan dan Saran.

Bab I Pendahuluan menjelaskan latar belakang masalah, rumusan masalah
berdasarkan latar belakang, tujuan Tugas Akhir, batasan masalah dari
implementasi tujuan Tugas Akhir, metodologi pengerjaan Tugas Akhir, dan
sistematika pembahasan laporan Tugas Akhir.

Bab II Studi literatur menjelaskan mengenai hasil studi literatur-literatur yang
dibutuhkan untuk menganalisis masalah dan mengimplementasikan solusi untuk
menyelesaikan masalah serta beberapa penelitian yang terkait mengenai Tugas
Akhir ini.

Bab III Deskripsi Solusi menjelaskan mengenai deskripsi umum permasalahan
\textit{capstone}, analisis masalah, analisis solusi, dan rancangan implementasi
solusi.

Bab IV Implementasi dan Evaluasi menjelaskan mengenai hasil eksplorasi kakas,
implementasi solusi yang dibuat, validasi hasil implementasi, distribusi hasil
implementasi, dan evaluasi implementasi.

Bab V Kesimpulan dan Saran menjelaskan mengenai kesimpulan dari pengerjaan Tugas
Akhir dan saran untuk pengembangan lanjut Tugas Akhir atau topik Tugas Akhir.
