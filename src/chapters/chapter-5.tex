\chapter{Kesimpulan dan Saran}

Bab ini membahas mengenai kesimpulan dan saran. Kesimpulan membahas mengenai
ketercapaian tujuan Tugas Akhir dan permasalahan yang diselesaikan dalam Tugas
Akhir. Saran membahas mengenai hal-hal yang dapat dikembangkan lebih lanjut.

\section{Kesimpulan}

Kesimpulan Tugas Akhir ini berdasarkan hasil implementasi, validasi, dan
evaluasi yang telah dilakukan adalah sebagai berikut:

\begin{enumerate}
	\item Trem dan angkot berhasil diimplementasi di dunia simulasi simulator
	CARLA. Trem dan angkot stabil beroperasi ketika simulasi dijalankan. Trem
	dan angkot diimplementasikan sebagai objek kendaraan roda 4.

	\item Sepeda onthel, sepeda motor, dan becak berhasil diimplementasikan di
	dunia simulasi simulator CARLA. Sepeda onthel, sepeda motor, dan becak
	diimplementasikan sebagai objek statik sehingga stabil ketika simulasi
	berjalan. Sepeda onthel, sepeda motor, dan becak diimplementasikan sebagai
	objek statik karena tidak stabil saat simulasi berjalan jika
	diimplementasikan sebagai objek kendaraan. Ketidakstabilan disebabkan oleh
	kesalahan penghubungan model 3D dengan \textit{armature}, kesalahan
	konfigurasi roda pada \textit{blueprint} kendaraan, dan dokumentasi yang
	kurang lengkap.

	\item Stasiun, rambu lalu lintas, dan rel berhasil diimplementasi di
	dunia simulasi simulator CARLA.

	% \item Distribusi hasil implementasi telah dibuat dalam bentuk CARLA versi
	% \textit{packaged}.

\end{enumerate}

\section{Saran}

Berikut adalah saran untuk pengembangan lanjut Tugas Akhir atau topik Tugas
Akhir:

\begin{enumerate}
	\item Membuat kota baru sehingga kendaraan berkendara di sisi sebelah kiri.
	\item Mengembangkan CARLA agar dapat melakukan implementasi kendaraan roda
	n dan kendaraan roda ganjil.
	\item Membuat implementasi kendaraan roda 2 yang stabil beserta dengan
	perbaikan dokumentasi.
	\item Melakukan migrasi aset yang telah dibuat ke versi CARLA terbaru
	sehingga dapat menggunakan fitur-fitur baru yang dikembangkan.
\end{enumerate}
