\chapter{Analisis dan Rancangan Implementasi Objek dan Lingkungan}

% Tujuan utama penulisan bab ini adalah untuk menguraikan rencana penyelesaian
% masalah tugas akhir yang akan dieksekusi secara utuh pada saat pelaksanaan
% Tugas Akhir II. Bab ini merupakan bab penutup Laporan Tugas Akhir I yang dapat
% dipandangsebagai bab yang akan menjembatani perpindahan ke proses pelaksanaan
% Tugas Akhir II. Pengembangan lebih lanjut dari bab ini dapat menjadi bagian
% dari bab Deskripsi Solusi pada Laporan Tugas Akhir.

Bab ini membahas analisis permasalah pada aplikasi simulasi yang sudah ada,
analisis terhadap objek dan lingkungan yang akan diimplementasi, analisis solusi
yang akan diterapkan untuk mengatasi masalah.

\section{Analisis Masalah Simulasi dan Aset Simulasi}
Simulasi trem otonom untuk saat ini telah diinisiasi namun baru sebatas
eksplorasi simulator CARLA dengan menambahkan beberapa kendaraan dan mengetes
kendaraan dalam simulasi. Aset simulasi untuk saat ini baru berupa aset bawaan
dari CARLA. Aset bawaan CARLA seperti peta kota, kendaraan, bangungan, rambu
lalu lintas, dan lain-lain merupakan aset yang mencerminkan kota-kota di Amerika
Serikat pada umumnya. Implementasi objek dan lingkungan Indonesia dibutuhkan
agar simulasi serupa dengan kehidupan nyata di Indonesia sehingga ketika
pengujian strategi kemudi trem otonom memiliki tingkat akurasi dan presisi yang
tinggi. Simulasi yang sesuai dengan keadaan aslinya sangat berpengaruh terhadap
sistem trem otonom.

Modul simulasi ini merupakan bagian dari pengembangan sistem otonom dengan
menggunakan kecerdasan buatan untuk trem. Modul ini bertujuan untuk memenuhi
kebutuhan pengujian virtual strategi kemudi trem otonom dengan simulasi.
Simulasi ini bertujuan untuk melakukan validasi strategi kemudi kecerdasan
buatan trem otonom yang telah dikembangkan.

\section{Analisis Solusi}

% 2. Analisis solusi dan rencana penyelesaian masalah
%% - Pilhan solusi, berdasarkan hasil studi literatur

%% - Analisis dan deskripsi pilihan solusi

%% - Langkah-langkah menuju solusi dan penjelasan

%% - Modul/subsistem/komponen yang akan dikembangkan (jelaskan)

%% - Alur umum algoritma / langkah-langkah pengembangan sistem (jelaskan)

%% - Kakas yang diperlukan

% Diagram blok bisa digunakan
% Bisa menggunakan lampiran
