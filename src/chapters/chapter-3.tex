\chapter{Analisis dan Rancangan Implementasi Objek dan Lingkungan}

% Tujuan utama penulisan bab ini adalah untuk menguraikan rencana penyelesaian
% masalah tugas akhir yang akan dieksekusi secara utuh pada saat pelaksanaan
% Tugas Akhir II. Bab ini merupakan bab penutup Laporan Tugas Akhir I yang dapat
% dipandangsebagai bab yang akan menjembatani perpindahan ke proses pelaksanaan
% Tugas Akhir II. Pengembangan lebih lanjut dari bab ini dapat menjadi bagian
% dari bab Deskripsi Solusi pada Laporan Tugas Akhir.

Bab ini membahas analisis permasalah pada aplikasi simulasi yang sudah ada,
analisis terhadap objek dan lingkungan yang akan diimplementasi, analisis solusi
yang akan diterapkan untuk mengatasi masalah.

\section{Analisis Masalah Simulasi dan Aset Simulasi}
Simulasi trem otonom untuk saat ini telah diinisiasi namun baru sebatas
eksplorasi simulator CARLA dengan menambahkan beberapa kendaraan dan mengetes
kendaraan dalam simulasi. Aset simulasi untuk saat ini baru berupa aset bawaan
dari CARLA. Aset bawaan CARLA seperti peta kota, kendaraan, bangungan, rambu
lalu lintas, dan lain-lain merupakan aset yang mencerminkan kota-kota di Amerika
Serikat pada umumnya. Implementasi objek dan lingkungan Indonesia dibutuhkan
agar simulasi serupa dengan kehidupan nyata di Indonesia sehingga ketika
pengujian strategi kemudi trem otonom memiliki tingkat akurasi dan presisi yang
tinggi. Simulasi yang sesuai dengan keadaan aslinya sangat berpengaruh terhadap
sistem trem otonom.

Modul simulasi ini merupakan bagian dari pengembangan sistem otonom dengan
menggunakan kecerdasan buatan untuk trem. Modul ini bertujuan untuk memenuhi
kebutuhan pengujian virtual strategi kemudi trem otonom dengan simulasi.
Simulasi ini bertujuan untuk melakukan validasi strategi kemudi kecerdasan
buatan trem otonom yang telah dikembangkan. Validasi diperlukan agar kecerdasan
buatan yang dikembangkan dapat beroperasi dengan lancar di lingkungan Indonesia.
Oleh karena itu, dibutuhkan aset simulasi yang sesuai dengan keadaan Indonesia.

\section{Analisis Solusi}
Permasalahan yang telah dibahas dapat diselesaikan dengan mengubah aset yang
sudah ada atau menambahkan aset baru. Aset seperti peta kota harus dibuat
langsung menggunakan CARLAUE4 atau menggunakan RoadRunner yang kemudian diimpor
ke dalam CARLAUE4. Aset baru yang lain dapat dibuat menggunakan aplikasi
\textit{3D modelling}. Aset dibuat secara lengkap dan rinci agar perilaku aset
sesuai dengan yang asli.

Aset yang telah jadi model 3D-nya selanjutnya dimasukkan ke dalam CARLAUE4.
Proses impor aset ini dilakukan dengan mengikuti panduan yang telah tersedia di
dokumentasi CARLA. Proses impor aset harus dilakukan dengan benar agar perilaku
aset baik dan stabil untuk simulasi. Terdapat aset khusus yang harus ditambahkan
ke dalam file 3D model kendaraan dan diatur ke model kendaraan. Aset khusus
tersebut merupakan aset \textit{bone} untuk \textit{rigging} roda kendaraan.
Setelah pengaturan aset dalam CARLAUE4 selesai, dilakukan verifikasi aset dengan
cara menjalankan simulasi dan mengamati cara kendaraan beroperasi. Implementasi
aset berupa kendaraan bus telah berhasil dilakukan pada penelitian lain
\parencite{related-work-xiang}. Proses implementasi kendaraan bus dilakukan
dengan membuat model 3D bus, mengimpor model 3D bus ke dalam CARLAUE4, mengedit
aset bus tersebut dalam CARLAUE4, dan memverifikasi hasil impor aset bus
tersebut.

% cara import aset yang lain (yang khusus, yang detail di bawah aja)
% adding details maybe

Implementasi objek dan lingkungan Indonesia yang akan dilakukan adalah dengan
menambahkan aset dan membuat aset baru. Implementasi yang menambahkan aset
meliputi kendaraan, rambu lalu lintas, dan lingkungan sederhana. Kendaraan yang
akan diimplementasikan adalah trem otonom, angkot, becak, gerobak, motor, dan
sepeda onthel. Trem akan dipasangkan berbagai sensor seperti Lidar, kamera RGB,
dan Radar untuk menangkap lingkungan sekitar.

% - Langkah-langkah menuju solusi dan penjelasan

% - Alur umum algoritma / langkah-langkah pengembangan sistem (jelaskan)
%   a. cara import/nambahin aset lain:
%      - props: signs, gedung?
%      - peta baru/buat?
%      - testing/validating asetnya udah masuk dan behave bener atau belum
%   b. cara nambahin sensor di trem
%   c. asset customization

%% - Kakas yang diperlukan
Pada Proses implementasi aset membutuhkan simulator CARLA dan CARLAUE4 yang
dibangun atau di-\textit{compile} sendiri bukan menggunakan program atau
\textit{binary} yang sudah di-\textit{compile}.
% Proses membangun kedua program ini membutuhkan program ...
Dibutuhkan aplikasi \textit{3D modelling} seperti Blender untuk membuat aset.
Aplikasi RoadRunner akan digunakan untuk membuat aset peta.

\section{Rancangan Implementasi Objek dan Lingkungan Indonesia di Simulator}
Proses implementasi objek dan lingkungan di simulator CARLA dilakukan dengan
langkah-langkah sebagai berikut.
\begin{enumerate}
	\item Membuat aset baru dengan Blender dan RoadRunner jika diperlukan.

	Aset dibuat mengikuti panduan yang sudah ada di dokumentasi CARLA.
	% Hal tersebut meliputi bentuk poligon, jumlah poligon, dan lain-lain.

	\item Melakukan impor dan edit aset ke dalam CARLAUE4.

	Aset yang ingin dimasukkan ke lingkungan simulasi dan diedit harus dilakukan
	dengan aplikasi CARLAUE4.

	% untuk import vehicles:
	% https://carla.readthedocs.io/en/0.9.13/tuto_A_add_vehicle/#bind-and-model-the-vehicle
	% untuk buat peta:
	% https://carla.readthedocs.io/en/0.9.13/tuto_M_generate_map/

	% links:
	% https://carla.readthedocs.io/en/0.9.13/tuto_A_add_props/
	% https://carla.readthedocs.io/en/0.9.13/tuto_A_material_customization/

	\item Melakukan validasi aset yang telah diimpor.

	Validasi aset dapat dilakukan dengan menjalankan simulator dan mengamati
	aset dalam simulasi. Validasi ini dilakukan untuk memastikan aset yang
	diimpor sudah sesuai dengan yang diinginkan dan stabil dalam simulasi.

	\item Melakukan ekspor aset agar dapat didistribusikan.

	Aset yang ingin digunakan oleh pengguna lain dapat diekspor sehingga
	\textit{portable}. Hal ini dapat dilakukan dengan menjalankan perintah di
	terminal.

	% untuk trem:
	% https://carla.readthedocs.io/en/0.9.13/tuto_G_retrieve_data/#spawn-the-ego-vehicle
\end{enumerate}

% jelasin per langkah
% - siapin aset
% - impor aset
