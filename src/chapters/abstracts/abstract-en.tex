\chapter*{Abstract}
\addcontentsline{toc}{chapter}{ABSTRACT}

\begin{center}
	\center
	\begin{doublespace}
		\Large \bfseries \MakeUppercase{\thetitleinenglish}

		\normalfont \normalsize
		Oleh

		\theauthor
	\end{doublespace}
\end{center}

\begin{singlespace}
	% TODO: write abstract
	Abstrak berisi ringkasan apa yang telah dikerjakan dalam tugas akhir. Ada
	beberapa hal yang perlu diperhatikan dalam penulisan abstrak. Pertama,
	abstrak harus memuat permasalahan yang dikaji, metode/teknik yang digunakan
	untuk menyelesaikan masalah, hasil yang dicapai / evaluasi kajian,
	kesimpulan yang diperoleh, dan kata kunci. Kedua, cara penulisannya harus
	padat dan terarah. Setiap kalimat harus dapat memberikan informasi sebanyak
	dan setepat mungkin, mudah dibaca dan dimengerti. Panjang ringkasan dibatasi
	maksimal 300 kata dan ditulis dengan satu spasi. Panjang ringkasan dibatasi
	maksimal 300 kata dan ditulis dengan satu spasi.

	Kata kunci: ringkasan, singkat, padat.
\end{singlespace}

\clearpage
