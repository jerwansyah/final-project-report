\chapter*{Abstract}
\addcontentsline{toc}{chapter}{ABSTRACT}

\begin{center}
	\center
	\begin{onehalfspace}
		\Large \bfseries \MakeUppercase{\thetitleinenglish}

		\normalfont \normalsize
		Oleh

		\theauthor
	\end{onehalfspace}
\end{center}

\begin{singlespace}
	The research "Development of an Autonomous System using Artificial
	Intelligence for Trams" focuses on developing an autonomous system for
	trams. This autonomous system's development requires simulation to
	facilitate its implementation, and the simulation uses a
	"software-in-the-loop" scheme with the CARLA simulator as the software tool.
	The simulation needs to be as close to the real world as possible to support
	the system's development. This final project blabal? the implementation of
	tram, local objects, and typical Indonesian environments in the CARLA
	simulator for autonomous tram simulation. After exploring the CARLA editor
	and its documentation, 3D model assets are created, those assets are then
	imported, the imported assets are edited, the edited assets are integrated
	into the simulation, and implementations are then validated. The implemented
	objects include trams, public transport vehicles ("angkot"), traditional
	bicycles ("sepeda onthel"), motorcycles, pedicabs ("becak"), train stations,
	traffic signs, and railway tracks. The results of the implementation are
	validated and evaluated to ensure that these assets or objects, particularly
	the vehicles, remain stable within the simulation and can follow both user
	control and autopilot control. The implementation results show that trams
	and angkots are stable as four-wheeled vehicles and can follow user control
	and autopilot control. On the other hand, traditional bicycles, motorcycles,
	and pedicabs remain stable as static objects and are not implemented as
	two-wheeled vehicles. Train stations, traffic signs, and railway tracks are
	stable as static objects as well. Sepeda onthel, motorcycles, and pedicabs
	are not implemented as two-wheeled vehicles because they are unstable during
	the simulation. This instability is caused by asset misconfiguration. The
	implemented assets are then exported, making them available for other
	autonomous system developers to use these assets.

	Keywords: autonomous tram, simulation, CARLA simulator, Indonesia local
	objects.
\end{singlespace}

\clearpage
