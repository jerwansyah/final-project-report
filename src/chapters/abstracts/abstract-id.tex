\chapter*{ABSTRAK}
\addcontentsline{toc}{chapter}{ABSTRAK}

\begin{center}
	\center
	\begin{doublespace}
		\Large \bfseries \MakeUppercase{\thetitle}

		\normalfont \normalsize
		Oleh

		\theauthor
	\end{doublespace}
\end{center}

\begin{singlespace}
	Penelitian ``Pengembangan Sistem Otonomi dengan Menggunakan Kecerdasan
	\textit{Artificial} untuk Trem"  merupakan penelitian yang mengembangkan
	sistem otonomi untuk trem. Pengembangan sistem otonomi ini membutuhkan
	simulasi untuk memudahkan pengembangan sistem tersebut. Simulasi
	menggunakan skema \textit{software-in-the-loop} dengan simulator CARLA
	sebagai perangkat lunaknya. Dibutuhkan lingkungan simulasi yang identik
	dengan lingkungan yang asli untuk mendukung pengembangan sistem. Tugas Akhir
	ini meneliti implementasi trem, objek lokal, dan lingkungan khas Indonesia
	dalam simulator CARLA untuk simulasi trem otonom. Setelah mengeksplorasi
	editor CARLA dan dokumentasinya, dilakukan pembuatan aset model 3D,
	pengimporan aset, pengeditan aset dalam editor, pengintegrasian aset ke
	dalam simulasi, dan validasi hasil implementasi. Objek yang diimplementasi
	adalah trem, angkot, sepeda onthel, sepeda motor, becak, stasiun, rambu lalu
	lintas, dan rel. Hasil implementasi divalidasi dan dievaluasi untuk
	memastikan aset-aset atau objek-objek tersebut stabil dalam simulasi. Hasil
	implementasi adalah: Trem dan angkot stabil sebagai kendaraan roda 4; Sepeda
	onthel, sepeda motor, dan becak stabil sebagai objek statik dan bukan
	sebagai kendaraan roda 2 atau 3; dan Stasiun, rambu lalu lintas, dan rel
	stabil sebagai objek statik. Hasil implementasi selanjutnya diekspor
	sehingga dapat digunakan dalam oleh berbagai pengembang sistem otonom.

	Kata kunci: trem otonom, simulasi, CARLA, \textit{software-in-the-loop
	simulation}.
\end{singlespace}

\clearpage
