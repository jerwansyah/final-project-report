\chapter*{ABSTRAK}
\addcontentsline{toc}{chapter}{ABSTRAK}

\begin{center}
	\center
	\begin{onehalfspace}
		\Large \bfseries \MakeUppercase{\thetitle}

		\normalfont \normalsize
		Oleh

		\theauthor
	\end{onehalfspace}
\end{center}

\begin{singlespace}
	Penelitian ``Pengembangan Sistem Otonomi dengan Menggunakan Kecerdasan
	\textit{Artificial} untuk Trem"  merupakan penelitian yang mengembangkan
	sistem otonomi untuk trem. Pengembangan sistem otonomi ini membutuhkan
	simulasi untuk memudahkan pengembangan sistem tersebut. Simulasi menggunakan
	skema \textit{software-in-the-loop} dengan simulator CARLA sebagai perangkat
	lunaknya. Dibutuhkan dunia simulasi yang identik dengan dunia nyata untuk
	mendukung pengembangan sistem. Tugas Akhir ini meneliti implementasi objek
	trem, objek lokal, dan lingkungan khas Indonesia dalam dunia simulasi
	simulator CARLA untuk simulasi trem otonom. Setelah mengeksplorasi editor
	CARLA dan dokumentasinya, dilakukan pembuatan aset model 3D, pengimporan
	aset, pengeditan aset dalam editor, pengintegrasian aset ke dalam simulasi,
	dan validasi hasil implementasi. Objek yang diimplementasi adalah trem,
	angkot, sepeda onthel, sepeda motor, becak, stasiun, rambu lalu lintas, dan
	rel. Hasil implementasi divalidasi dan dievaluasi untuk memastikan aset-aset
	atau objek-objek tersebut terutama kendaaraan stabil dalam simulasi dan
	dapat mengikuti kendali pengguna dan kendali \textit{autopilot}. Hasil
	implementasi adalah: Trem dan angkot stabil sebagai kendaraan roda 4 dan
	dapat mengikut kendali pengguna dan kendali \textit{autopilot}; Sepeda
	onthel, sepeda motor, dan becak stabil sebagai objek statik dan bukan
	sebagai kendaraan roda 2; dan Stasiun, rambu lalu lintas, dan rel stabil
	sebagai objek statik. Sepeda onthel, sepeda motor, dan becak tidak
	diimplementasikan sebagai kendaraan roda 2 karena tidak stabil ketika
	simulasi berjalan. Hal tersebut disebabkan oleh kesalahan konfigurasi aset.
	Hasil implementasi selanjutnya diekspor sehingga dapat digunakan dalam oleh
	berbagai pengembang sistem otonom.

	Kata kunci: trem otonom, simulasi, simulator CARLA, objek lokal Indonesia.
\end{singlespace}

\clearpage
