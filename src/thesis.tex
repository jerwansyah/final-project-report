%--------------------------------------------------------------------%
%
% Berkas utama templat LaTeX.
%
% author Petra Barus, Peb Ruswono Aryan, Faris Rizki Ekananda
%
%--------------------------------------------------------------------%
%
% Berkas ini berisi struktur utama dokumen LaTeX yang akan dibuat.
%
%--------------------------------------------------------------------%

\documentclass[bahasa, 12pt, a4paper, onecolumn, oneside, final]{report}

%-------------------------------------------------------------------%
%
% Konfigurasi dokumen LaTeX untuk laporan tesis IF ITB
%
% @author Petra Novandi
%
%-------------------------------------------------------------------%
%
% Berkas asli berasal dari Steven Lolong
%
%-------------------------------------------------------------------%

% Ukuran kertas
\special{papersize=210mm,297mm}

% Setting margin
\usepackage[top=3cm,bottom=3cm,left=4cm,right=3cm]{geometry}

\usepackage{mathptmx}

% Judul bahasa Indonesia
\usepackage[bahasa]{babel}

% Format citation
\usepackage[style=apa,backend=biber]{biblatex}

\usepackage[utf8]{inputenc}
\usepackage{graphicx}
\usepackage{titling}
\usepackage{blindtext}
\usepackage{sectsty}
\usepackage{chngcntr}
\usepackage{etoolbox}
\usepackage{titlesec}       % Package Format judul
\usepackage{titletoc}       % Package Format judul di toc
\usepackage{tocbibind}      % Package untuk masukkan toc, lot, lof ke Daftar Isi
\usepackage{scrwfile}       % Package untuk membuat Daftar Lampiran dari toc
\usepackage{parskip}
\usepackage{afterpage}
\usepackage{relsize}
\usepackage{setspace}
\usepackage{subfig}
\usepackage{longtable}
% \usepackage{pgfgantt}       % Package untuk membuat Gantt Chart
% \usepackage{lscape}         % Package untuk membuat landscape
\usepackage{listings}       % Package lstlistings and stuff
\usepackage{etoolbox}% http://ctan.org/pkg/etoolbox

\graphicspath{{resources/}}   % letak direktori penyimpanan gambar

% Setting daftar lampiran
\newcommand*{\lopname}{DAFTAR LAMPIRAN}
\TOCclone[\lopname]{toc}{atoc}
\addtocontents{atoc}{\protect\value{tocdepth}=-1}
\newcommand\listofappendices{
  \cleardoublepage
  \phantomsection
  \listofatoc
  \addcontentsline{toc}{chapter}{\lopname}
}

\newcommand*\savedtocdepth{}
\AtBeginDocument{%
  \edef\savedtocdepth{\the\value{tocdepth}}%
}

\let\originalappendix\appendix
\renewcommand\appendix{%
  \originalappendix
  \cleardoublepage
  \addtocontents{toc}{\protect\value{tocdepth}=-1}%
  \addtocontents{atoc}{\protect\value{tocdepth}=\savedtocdepth}%

  \renewcommand{\thesection}{\thechapter.\Alph{section}}

  \titlecontents{chapter}
    [0pt]
    {\bfseries}
    {Lampiran \thecontentslabel.\quad}
    {}
    {\hfill\contentspage}

  \titleformat{\chapter}[block]
    {\bfseries}
    {\chaptertitlename\ \thechapter.\quad}{0pt}
    {\bfseries}
}

% Setting titik pada toc
\makeatletter
\renewcommand{\@dotsep}{1}
\patchcmd{\@chapter}{\addtocontents{lof}{\protect\addvspace{10\p@}}}{}{}{}
\makeatother

\usepackage[hidelinks]{hyperref}       % Package untuk link di daftar isi. Ubah jadi \usepackage[hidelinks]{hyperref} apabila ingin menghilangkan kotak merah disekitar link

% Setel title pada chapter-chapter di toc, lof, lot
\titlecontents{chapter}
  [0pt]
  {\bfseries}
  {\MakeUppercase{Bab} \thecontentslabel\quad\uppercase}
  {}
  {\mdseries\titlerule*[0.35em]{.}\bfseries\contentspage}
\titlecontents{figure}
  [0pt]
  {}
  {Gambar \thecontentslabel.\quad}
  {}
  {\mdseries\titlerule*[0.35em]{.}\contentspage}
\titlecontents{table}
  [0pt]
  {}
  {Tabel \thecontentslabel.\quad}
  {}
  {\mdseries\titlerule*[0.35em]{.}\contentspage}

% Masukin Daftar Pustaka ke toc
\let\originalprintbibliography\printbibliography
\renewcommand\printbibliography{%
  \phantomsection
  \cleardoublepage
  \originalprintbibliography
  \addcontentsline{toc}{chapter}{\bibname}
}

% Line satu setengah spasi
\renewcommand{\baselinestretch}{1.5}

% Setting judul
\chapterfont{\centering \Large}
\titleformat{\chapter}[display]
  {\Large\centering\bfseries}
  {\chaptertitlename\ \thechapter}{0pt}
    {\Large\bfseries\uppercase}

% Setting nomor pada subbsubsubbab
\setcounter{secnumdepth}{3}

% Setting kedalaman pada toc
\setcounter{tocdepth}{3}

\makeatletter

\makeatother

% Counter untuk figure dan table.
% \counterwithin{figure}{section}
% \counterwithin{table}{section}

% Define blank page
\newcommand*{\blankpage}{\afterpage{\null\newpage}}

% Remove Hypenation
\hyphenpenalty=10000
\exhyphenpenalty=10000
\sloppy % Makes TeX obey margins by stretching inter word spaces


\makeatletter
\setlength{\@fptop}{0pt}
\makeatother

\addbibresource{references.bib}

\begin{document}

% Basic configuration
% date and times
% TODO: change the date accordingly
\newcommand{\thedate}{\date{11 Juli 2023}}
\newcommand{\monthyear}{Juli 2023}

% titles
\newcommand{\capstonetitle}{Pengembangan Sistem Simulasi \textit{Autonomous Tram} dengan Simulator CARLA}
\newcommand{\thetitle}{Implementasi Objek Lokal dan Lingkungan Indonesia untuk
Simulasi Trem Otonom Menggunakan Simulator CARLA}
\newcommand{\thetitleinenglish}{Implementation of Indonesian Local Objects and
Enviromental Objects for Autonomous Tram Simulation Using CARLA Simulator}

\title{\thetitle}

\newcommand{\subtitle}{
    \bfseries \Large
    Laporan Tugas Akhir - Capstone

    \capstonetitle
}

% identity
\newcommand{\authorname}{Jeane Mikha Erwansyah}
\newcommand{\authornim}{13519116}
\author{
    \authorname \\
    NIM : \authornim
}

% \newcommand{\advisoronename}{Achmad Imam Kistijantoro, S.T., M.Sc., Ph.D.}
% \newcommand{\advisoronenip}{197308092006041001}
% \newcommand{\advisortwoname}{Prof. Dr. Ir. Bambang Riyanto Trilaksono}
% \newcommand{\advisortwonip}{196211151987031004}

% advisors
\newcommand{\advisoronename}{Achmad Imam Kistijantoro, Ph.D.}
\newcommand{\advisoronenip}{197308092006041001}
\newcommand{\advisortwoname}{Prof. Dr. Bambang Riyanto Trilaksono}
\newcommand{\advisortwonip}{196211151987031004}
\newcommand{\advisorapproval}{
    \centering
    \normalsize \normalfont
    % commented out because of differences between capstone-team and approval-2
    % Bandung, \thedate \\
    % Mengetahui,

    \vspace{0.5cm}
    \setlength{\tabcolsep}{12pt}
    \begin{tabular}{c@{\hskip 0.5cm}c}
        Pembimbing I,               & Pembimbing II,             \\
                                    &                            \\
                                    &                            \\
                                    &                            \\
                                    &                            \\
        \underline{\advisoronename} & \underline{\advisortwoname} \\
        NIP. \advisoronenip     & NIP. \advisortwonip     \\
    \end{tabular}
    }

\pagenumbering{roman}
\setcounter{page}{1}

\input{chapters/cover}
\clearpage
\pagestyle{empty}

\begin{center}
    \smallskip

    \Large \bfseries \MakeUppercase{\thetitle}
    \vfill

    \subtitle
    \vfill

    \large Oleh

    \Large \theauthor

    \large Program Studi Teknik Informatika \\

    \normalsize \normalfont
    Sekolah Teknik Elektro dan Informatika \\
    Institut Teknologi Bandung

    % below is needed to fix the \thedate command not showing the first time
    \thedate
    \vfill
    % TODO: hapus kata draf
    Telah disetujui dan disahkan sebagai draf Laporan Tugas Akhir \\
    di Bandung, pada tanggal \thedate

    \advisorapproval

\end{center}
\clearpage

\clearpage
\pagestyle{empty}

\begin{center}
	\smallskip

	\Large \bfseries \MakeUppercase{
		Lembar Identitas \\
		Tugas Akhir Capstone
	}
	\vspace{0.5cm}

	\raggedright
	\begin{table}[h!]
		\large \bfseries
		\begin{onehalfspace}
		\begin{tabular}{p{0.3\textwidth} p{0.63\textwidth}}
			Judul Proyek TA : & \capstonetitle \\
		\end{tabular}
		\end{onehalfspace}
	\end{table}

	\normalsize \normalfont

	Anggota tim dan pembagian peran:

	\begin{table}[h!]
		\begin{onehalfspace}
		\begingroup
		\def\arraystretch{1.25}
		\begin{tabular}{|p{0.05\textwidth} | p{0.13\textwidth} | p{0.19\textwidth} | p{0.50\textwidth}|}
			\hline
			\textbf{No.} & \textbf{NIM} & \textbf{Nama}         & \textbf{Peran} \\
			\hline
			1.           & 13519116     & Jeane Mikha Erwansyah & \thetitle \\
			\hline
			2.           & 13519164     & Josep Marcello        & Pengembangan Simulasi \textit{Hardware-in-the-Loop} Kendaraan Otonom yang Menggunakan CARLA dan NVIDIA Driveworks \\
			\hline
			3.           & 13519188     & Jeremia Axel Bachtera & Pembangunan Modul Pengujian untuk Simulasi Trem Otonom Simulator CARLA \\
			\hline
		\end{tabular}
		\endgroup
		\end{onehalfspace}
	\end{table}

	\vfill
	\begin{center}
		\normalsize \normalfont
		Bandung, \thedate \\
		Mengetahui,
	\end{center}
	\advisorapproval

\end{center}
\clearpage


\pagestyle{plain}

\titleformat*{\section}{\centering\bfseries\Large\MakeUpperCase}
\titlespacing*{\chapter}{0pt}{0pt}{3pc}

\input{chapters/statement}
\chapter*{ABSTRAK}
\addcontentsline{toc}{chapter}{ABSTRAK}

\begin{center}
	\center
	\begin{onehalfspace}
		\Large \bfseries \MakeUppercase{\thetitle}

		\normalfont \normalsize
		Oleh

		\theauthor
	\end{onehalfspace}
\end{center}

\begin{singlespace}
	Penelitian ``Pengembangan Sistem Otonomi dengan Menggunakan Kecerdasan
	\textit{Artificial} untuk Trem"  merupakan penelitian yang mengembangkan
	sistem otonomi untuk trem. Pengembangan sistem otonomi ini membutuhkan
	simulasi untuk memudahkan pengembangan sistem tersebut. Simulasi menggunakan
	skema \textit{software-in-the-loop} dengan simulator CARLA sebagai perangkat
	lunaknya. Dibutuhkan dunia simulasi yang identik dengan dunia nyata untuk
	mendukung pengembangan sistem. Tugas Akhir ini meneliti implementasi objek
	trem, objek lokal, dan lingkungan khas Indonesia dalam dunia simulasi
	simulator CARLA untuk simulasi trem otonom. Setelah mengeksplorasi editor
	CARLA dan dokumentasinya, dilakukan pembuatan aset model 3D, pengimporan
	aset, pengeditan aset dalam editor, pengintegrasian aset ke dalam simulasi,
	dan validasi hasil implementasi. Objek yang diimplementasi adalah trem,
	angkot, sepeda onthel, sepeda motor, becak, stasiun, rambu lalu lintas, dan
	rel. Hasil implementasi divalidasi dan dievaluasi untuk memastikan aset-aset
	atau objek-objek tersebut terutama kendaaraan stabil dalam simulasi dan
	dapat mengikuti kendali pengguna dan kendali \textit{autopilot}. Hasil
	implementasi adalah: Trem dan angkot stabil sebagai kendaraan roda 4 dan
	dapat mengikut kendali pengguna dan kendali \textit{autopilot}; Sepeda
	onthel, sepeda motor, dan becak stabil sebagai objek statik dan bukan
	sebagai kendaraan roda 2; dan Stasiun, rambu lalu lintas, dan rel stabil
	sebagai objek statik. Sepeda onthel, sepeda motor, dan becak tidak
	diimplementasikan sebagai kendaraan roda 2 karena tidak stabil ketika
	simulasi berjalan. Hal tersebut disebabkan oleh kesalahan konfigurasi aset.
	Hasil implementasi selanjutnya diekspor sehingga dapat digunakan dalam oleh
	berbagai pengembang sistem otonom.

	Kata kunci: trem otonom, simulasi, simulator CARLA, objek lokal Indonesia.
\end{singlespace}

\clearpage

% \chapter*{Abstract}
\addcontentsline{toc}{chapter}{ABSTRACT}

\begin{center}
	\center
	\begin{doublespace}
		\Large \bfseries \MakeUppercase{\thetitleinenglish}

		\normalfont \normalsize
		Oleh

		\theauthor
	\end{doublespace}
\end{center}

\begin{singlespace}
	% TODO: write abstract
	Abstrak berisi ringkasan apa yang telah dikerjakan dalam tugas akhir. Ada
	beberapa hal yang perlu diperhatikan dalam penulisan abstrak. Pertama,
	abstrak harus memuat permasalahan yang dikaji, metode/teknik yang digunakan
	untuk menyelesaikan masalah, hasil yang dicapai / evaluasi kajian,
	kesimpulan yang diperoleh, dan kata kunci. Kedua, cara penulisannya harus
	padat dan terarah. Setiap kalimat harus dapat memberikan informasi sebanyak
	dan setepat mungkin, mudah dibaca dan dimengerti. Panjang ringkasan dibatasi
	maksimal 300 kata dan ditulis dengan satu spasi. Panjang ringkasan dibatasi
	maksimal 300 kata dan ditulis dengan satu spasi.

	Kata kunci: ringkasan, singkat, padat.
\end{singlespace}

\clearpage

\chapter*{Kata Pengantar}
\addcontentsline{toc}{chapter}{KATA PENGANTAR}

Puji syukur penulis panjatkan kepada Tuhan Yang Maha Esa atas segala berkat dan
rahmat-Nya sehingga penulis dapat menyelesaikan Tugas Akhir ini dengan judul
``\thetitle''. Penulis ingin mengucapkan terima kasih kepada semua pihak yang
telah mendukung dan membantu penulis dalam menyelesaikan Tugas Akhir ini.
Oleh karena itu, penulis menyampaikan ucapan terima kasih kepada:

\begin{enumerate}
	\item Bapak \advisoronename\verb| |dan Bapak \advisortwoname\verb| |selaku
	dosen pembimbing tugas Akhir yang telah memberikan bimbingan, arahan, dan
	masukan penulis selama pengerjaan Tugas Akhir ini.
	\item Bapak Handoko Supeno, S.T., M.T. sebagai pemimpin tim simulasi yang
	telah memberikan bimbingan, arahan, dan masukan kepada penulis dalam
	pengembangan pengerjaan Tugas Akhir ini dan selama pengerjaan laporan.
	\item Keluarga penulis yang telah menemani penulis dan memberikan dukungan
	serta doa untuk penulis.
	\item Anggota tim \textit{capstone}, yaitu Josep Marcello dan Jeremia Axel
	yang telah berjuang bersama dalam pengembangan simulasi trem otonom dan
	penyelesaian Tugas Akhir.
	\item Seluruh dosen Institut Teknologi Bandung yang telah mengajarkan berbagai
	ilmu selama penulis menempuh perkuliahan.
	\item Teman-teman penulis yang telah menemani penulis, memberikan dukungan,
	memberikan semangat, dan meluangkan waktu untuk bermain terutama Amara,
	Alex, Suggoi, Nathan, dan anggota keluarga yee lainnya.
\end{enumerate}

% \begin{flushright}
% 	Bandung, \thedate \\
% 	\vspace{15mm}

% 	Penulis
% \end{flushright}

\clearpage



% Setting judul toc, lot, lof, bib
\renewcommand{\contentsname}{DAFTAR ISI}
\renewcommand{\listfigurename}{DAFTAR GAMBAR}
\renewcommand{\listtablename}{DAFTAR TABEL}
\renewcommand{\bibname}{DAFTAR PUSTAKA}

% \renewcommand{\cftchapleader}{\cftdot} % for chapters

\tableofcontents
% \listofappendices
\listoffigures
% \listoftables
\chapter*{Daftar Istilah}
\addcontentsline{toc}{chapter}{DAFTAR ISTILAH}

\begingroup
\def\arraystretch{1.25}
\begin{onehalfspace}
\begin{longtable}{p{0.3\textwidth} p{0.66\textwidth}}

	\textit{Blueprint} & Aset untuk mengintegrasikan aset atau aktor lain ke dalam lingkungan simulasi dengan konfigurasi yang diperlukan aset atau aktor tersebut \\
	\textit{Ego vehicle} & Kendaraan yang dikendalikan pengguna dalam simulasi \\
	\textit{Factory} & Desain pola program yang berfungsi untuk menghasilkan atau memunculkan hal-hal tertentu \\
	FBX & Format berkas aset digital \textit{proprietary} yang meliputi model 3D, animasi, tekstur, dan lain-lain \\
	\textit{Hardware-in-the-Loop} & Penggunaan perangkat keras untuk menguji atau mengevaluasi interaksi dan kinerja perangkat keras dan komponen lain yang terhubung dengan perangkat keras tersebut \\
	Kurva Bézier & Teknik grafis komputer untuk membuat dan merepresentasikan kurva \\
	\textit{Localization} & Penentuan lokasi dari data sensor di sebuah lingkungan \\
	\textit{Mesh} & Objek yang terdiri atas kumpulan titik, sisi, dan permukaan \\
	\textit{Mixed traffic} & Lalu lintas yang melibatkan berbagai kendaraan dan/atau pihak \\
	\textit{Software-in-the-Loop} & Pengujian dan/atau validasi perangkat lunak yang terpisah dengan perangkat keras \\
	\textit{Spline} & Kumpulan kurva yang membentuk satu kurva panjang \\
	Trem & Kereta transportasi umum yang berjalan di atas rel di jalan raya \\
	\textit{Vertices group} & Kumpulan titik yang berguna untuk membatasi operasi ke area spesifik pada \textit{mesh} \\

\end{longtable}
\end{onehalfspace}
\endgroup
\clearpage

\chapter*{Daftar Singkatan}
\addcontentsline{toc}{chapter}{DAFTAR SINGKATAN}

\begingroup
\def\arraystretch{1.25}
\begin{tabular}{p{4cm}l}
	API & \textit{Application Programming Interface} \\
	CARLA & Car Learning to Act \\
	CARLAUE4 & CARLA Unreal Engine 4 \\
	GPS & \textit{Global Positioning System} \\
	NPC & \textit{Non-Player Character} \\
	UE4 & Unreal Engine 4 \\

	% AI & \textit{Artificial Intelligence} \\
	% ML & \textit{Machine Learning} \\

\end{tabular}
\endgroup
\clearpage


\newpage

\titleformat*{\section}{\bfseries\large}
\pagenumbering{arabic}

%----------------------------------------------------------------%
% Konfigurasi Bab
%----------------------------------------------------------------%
\setcounter{page}{1}
% TODO: find a better way to set counter
\setcounter{section}{1}
\renewcommand{\chaptername}{BAB}
\renewcommand{\thechapter}{\Roman{chapter}}
%----------------------------------------------------------------%

%----------------------------------------------------------------%
% Dafter Bab
% Untuk menambahkan daftar bab, buat berkas bab misalnya `chapter-6` di direktori `chapters`, dan masukkan ke sini.
%----------------------------------------------------------------%
\chapter{Pendahuluan}

% Bab Pendahuluan secara umum yang dijadikan landasan kerja dan arah kerja
% penulis Tugas Akhir, berfungsi mengantar pembaca untuk membaca laporan tugas
% akhir secara keseluruhan.

\section{Latar Belakang}

Perkembangan teknologi yang pesat memungkinkan terjadinya perkembangan teknologi
pada industri otomotif. Salah satu inovasi teknologi otomotif adalah kendaraan
yang dapat berjalan sendiri secara otonom tanpa adanya pengemudi manusia.
Kendaraan otonom tersebut dapat berkemudi sendiri dengan adanya kecerdasan
buatan. Teknologi kendaraan otonom ini tidak terbatas pada kendaraan pribadi
saja, tetapi dapat diaplikasikan juga pada kendaraan umum seperti trem.

Pengembangan kendaraan otonom untuk penggunaan secara komersial perlu dilakukan
dengan teliti sehingga kendaraan otonom tersebut dapat beroperasi dengan baik,
dapat diandalkan, dapat nyaman digunakan, dapat meminimalisasikan biaya bahan
bakar karena menggunakan listrik, dan dapat memimalisasi kecelakaan lalu lintas.
Kendaraan otonom yang diharapkan tersebut membutuhkan pengembangan kecerdasan
buatan yang dapat mengoperasikan kendaraan yang spesifik. Oleh karena itu,
pengembangan kendaraan otonom membutuhkan biaya yang besar dan waktu yang lama
sebab dibutuhkan data yang banyak dan bervariasi untuk melatih kecerdasan
buatan. Selain itu, dibutuhkan juga pengujian kendaraan otonom untuk melakukan
validasi kecerdasan buatan yang telah dibuat. Pengumpulan data untuk pelatihan
dapat membutuhkan waktu yang lama sebab terdapat banyak faktor eksternal yang
tidak dapat diprediksi atau dikontrol yang membuat data yang dikumpulkan tidak
bervariasi. Proses untuk mencari data yang bervariasi dapat memakan waktu juga
sebab harus menunggu saat yang tepat untuk pengumpulan data. Selain pengumpulan
data, pengujian kendaraan otonom juga membutuhkan waktu yang lama dan biaya yang
besar sebab untuk memastikan kendaraan otonom dapat beroperasi dengan baik,
pengujian harus dilakukan berulang-ulang dan dalam berbagai skenario yang kadang
tidak dapat diatur sesuai dengan keinginan di saat yang diinginkan.

Pelatihan dan pengujian kendaraan otonom yang mahal ini dapat diatasi dengan
melakukannya secara virtual menggunakan aplikasi simulasi. Pelatihan dan
pengujian dengan menggunakan simulasi dapat menghemat biaya dan waktu karena
aktor, lingkungan, dan parameter simulasi untuk pengujian dapat diatur sesuai
dengan keinginan relatif lebih mudah dan lebih cepat. Lingkungan simulasi yang
identik dengan lingkungan nyata diperlukan sehingga kecerdasan buatan kendaraan
otonom cocok digunakan untuk lingkungan pada kenyataan. Oleh karena itu,
dibutuhkan implementasi aktor, objek, dan lingkungan simulasi.

Tugas Akhir ini membahas implementasi objek trem, implementasi objek lalu lintas
khas Indonesia, dan implementasi lingkungan simulasi yang serupa dengan
Indonesia menggunakan simulator Car Learning to Act (CARLA). Aset bawaan dari
CARLA seperti kendaraan, jalan, kota, lingkungan, dan aset lainnya mencerminkan
objek dan lingkungan di Amerika Serikat pada umumnya. Oleh karena itu, perlu
dibuat aset yang mencerminkan objek dan lingkungan di Indonesia. Lalu lintas
\textit{default} di CARLA adalah jalur kanan. Hal tersebut juga perlu
disesuaikan agar sama seperti lalu lintas di Indonesia.

\section{Rumusan Masalah}

Berdasarkan latar belakang yang telah dibahas, rumusan masalah yang akan dibahas
dalam Tugas Akhir ini adalah sebagai berikut.

\begin{enumerate}

	\item Bagaimana cara mengimplementasikan objek trem, objek lokal khas
	Indonesia di lingkungan simulator CARLA?

	\item Bagaimana cara mengimplementasikan lingkungan simulasi yang serupa
	dengan lingkungan Indonesia di simulator CARLA?

\end{enumerate}

\section{Tujuan}

Tujuan dari Tugas Akhir ini adalah untuk mengimplentasikan objek trem, objek
lokal khas Indonesia, dan lingkungan khas Indonesia di simulator CARLA untuk
memudahkan validasi algortima \textit{decision making} trem otonom.

\section{Batasan Masalah}

Penelititan untuk Tugas Akhir ini memiliki batasan masalah yaitu
pengimplementasian objek trem, objek lokal khas Indonesia, dan lingkungan
simulasi hanya terbatas pada lingkungan kota Solo. Kota Solo dipilih sebagai
lingkungan pengujian trem otonom karena di Solo sudah terdapat trem konvensional
yang beroperasi. Pengambilan data untuk pelatihan telah dan sedang dilakukan di
Solo. Pengujian trem otonom di luar simulasi pun akan dilakukan di Solo.
Tingkat kemiripan simulasi terbatas pada aplikasi simulator CARLA.

\section{Metodologi}

Metodologi yang digunakan dalam pengerjaan Tugas Akhir ini adalah sebagai
berikut.

\begin{enumerate}

	\item Analisis masalah

	Melakukan analisis terhadap rumusan masalah mengenai simulasi trem otonom.
	Analisis dilakukan dengan melakukan studi literatur mengenai simulasi
	kendaraan otonom.

	\item Eksplorasi kakas untuk mengimplementasikan solusi

	Melakukan eksplorasi kakas yang akan digunakan untuk implementasi dan
	validasi untuk menyelesaikan masalah yang telah dirumuskan.

	\item Implementasi solusi

	Melakukan implementasi objek trem, objek lokal khas Indonesia, dan
	lingkungan simulasi di \textit{editor} simulator CARLA (CARLA Unreal Engine
	4 Editor) agar dapat digunakan di simulator CARLA.

	\item Validasi dan analisis implementasi solusi

	Pada tahap ini dilakukan validasi dan analisis solusi yang sudah
	implementasi. Tahap ini dilakukan untuk memastikan bahwa objek trem, objek
	lokal khas Indonesia, dan lingkungan simulasi yang diimplementasikan
	berperilaku normal, stabil untuk digunakan dalam simulasi, dan dapat
	memenuhi kebutuhan simulasi.

\end{enumerate}

% \section{Jadwal Pelaksanaan Tugas Akhir}

% Berikut adalah jadwal pelaksanaan Tugas Akhir I perminggunya.

% \begin{figure}[ht]
% 	\linespread{0.8}
% 	\resizebox{\textwidth}{!}{
% 		\begin{ganttchart}[
% 				y unit title=0.5cm,
% 				y unit chart=1.3cm,
% 				bar height=0.7,
% 				vgrid,
% 				title height=1,
% 				bar label font=\tiny,
% 				bar label node/.style={
% 						text width=4cm,
% 						align=right,
% 						anchor=east,
% 						font=\raggedleft
% 					},
% 			]{1}{20}
% 			%labels
% 			\gantttitle{September}{4}
% 			\gantttitle{Oktober}{4}
% 			\gantttitle{November}{4}
% 			\gantttitle{Desember}{4}
% 			\gantttitle{Januari}{4} \\
% 			\gantttitlelist{1,...,4}{1}
% 			\gantttitlelist{1,...,4}{1}
% 			\gantttitlelist{1,...,4}{1}
% 			\gantttitlelist{1,...,4}{1}
% 			\gantttitlelist{1,...,4}{1} \\

% 			% tasks
% 			\ganttbar{Mempelajari Dokumentasi CARLA}{4}{4} \\
% 			\ganttbar{Eksplorasi CARLA}{5}{7} \\
% 			\ganttbar{Implementasi Objek Angkot}{8}{10} \\
% 			\ganttbar{Implementasi Objek Tram}{11}{13} \\
% 			\ganttbar{Implementasi Objek Kendaraan}{14}{16} \\
%             \ganttbar{Implementasi Lingkungan Baru}{17}{20} \\
%             \ganttbar{Sidang Tugas Akhir 1}{17}{18} \\
% 		\end{ganttchart}
% 	}
% 	% \caption{Jadwal Pelaksanaan Tugas Akhir (bagian 1)}
% 	\caption{Jadwal Pelaksanaan Tugas Akhir}
% \end{figure}

% % TODO
% % \begin{figure}[ht]
% % 	\begin{center}
% % 		\linespread{0.8}
% % 		\resizebox{\textwidth}{!}{
% % 			\begin{ganttchart}[
% % 					y unit title=0.5cm,
% % 					y unit chart=1.3cm,
% % 					bar height=0.7,
% % 					vgrid,
% % 					title height=1,
% % 					bar label font=\tiny,
% % 					bar label node/.style={
% % 							text width=4cm,
% % 							align=right,
% % 							anchor=east,
% % 							font=\raggedleft
% % 						},
% % 				]{1}{24}
% % 				%labels
% % 				\gantttitle{Februari}{4}
% % 				\gantttitle{Maret}{4}
% % 				\gantttitle{April}{4}
% % 				\gantttitle{Mei}{4}
% % 				\gantttitle{Juni}{4}
% % 				\gantttitle{Juli}{4} \\
% % 				\gantttitlelist{1,...,4}{1}
% % 				\gantttitlelist{1,...,4}{1}
% % 				\gantttitlelist{1,...,4}{1}
% % 				\gantttitlelist{1,...,4}{1}
% % 				\gantttitlelist{1,...,4}{1}
% % 				\gantttitlelist{1,...,4}{1} \\

% % 				% tasks
% % 				\ganttbar{Perancangan kerangka kerja pengujian}{1}{8} \\
% % 				\ganttbar{Implementasi kerangka kerja pengujian}{9}{16} \\
% % 			\end{ganttchart}
% % 		}
% % 	\end{center}
% % 	\caption{Jadwal Pelaksanaan Tugas Akhir (bagian 2)}
% % \end{figure}

\section{Sistematika Pembahasan}
% Subbab ini berisi penjelasan ringkas isi per bab. Penjelasan ditulis satu
% paragraf per bab buku.
Sistematika pembahasan laporan Tugas Akhir ini adalah sebagai berikut.

\begin{enumerate}

	\item Bab I Pendahuluan menjelaskan latar belakang masalah, rumusan masalah
	berdasarkan latar belakang, tujuan Tugas Akhir, batasan masalah dari
	implementasi tujuan, metodologi pengerjaan Tugas Akhir, dan sistematika
	pembahasan laporan Tugas Akhir.

	\item Bab II Studi literatur menjelaskan mengenai hasil studi yang
	dibutuhkan untuk menganalisis masalah dan mengimplementasikan solusi untuk
	menyelesaikan masalah. Penelitian terkait mengenai Tugas Akhir ini.

	\item Bab III Analisis dan Rancangan Implementasi Objek dan Lingkungan
	menjelaskan analisis masalah, analisis solusi, dan rancangan implementasi
	solusi.

	\item Bab IV Implementasi Objek dan Lingkungan menjelaskan implementasi
	solusi yang dibuat.

	\item Bab V Kesimpulan dan Saran menjelaskan mengenai kesimpulan dan saran
	pengerjaan tugas akhir dan penyusunan laporan tugas akhir.

\end{enumerate}

% \blankpage
\chapter{Studi Literatur}

Bab ini membahas hasil studi literatur yang diperlukan untuk melakukan
penyusunan Tugas Akhir ini. Bab ini membahas tentang kendaraan otonom, simulator
CARLA, Unreal Engine, Blender, dan penelitian terkait.
%RoadRunner,

\section{Kendaraan Otonom}

Kendaraan otonom atau \textit{Autonomous Vehicle} merupakan kendaraan yang
dirancang untuk memonitor keadaan jalan dan berkemudi sendiri tanpa bantuan
pengemudi. Kendaraan otonom dibuat dengan tujuan mengurangi jumlah kecelakaan,
mengurangi penggunaan energi, mengurangi polusi, mengurangi kemacetan, dan
meningkatkan akses transportasi \parencite{av-bagloee}. Trem otonom sangat mirip
dengan mobil otonom, keduanya beroperasi di lingkungan yang sama dan
berinteraksi dengan objek-objek yang sama pula. Perbedaan keduanya adalah trem
otonom terikat dengan rel trem sehingga trem tidak dapat berbelok keluar dari
rel untuk menghindari objek yang berpotensi menjadi halangan ataupun berhenti
mendadak karena terbatasi secara fisik dan berhenti mendadak dapat mencelakai
penumpang \parencite{at-palmer}.

Trem otonom secara umum memiliki perangkat keras berupa kendaraan fisik,
komputer, dan berbagai sensor. Komputer pada tram digunakan untuk mengoperasikan
trem secara otonom. Komputer juga memiliki kartu grafis untuk melakukan deteksi
objek hasil deteksi dari sensor-sensor di trem. Sensor-sensor tersebut adalah,
kamera, LIDAR, dan radar. Sedangkan pada sisi perangkat lunak, trem otonom pada
umumnya memiliki aplikasi untuk mengolah data sensor \parencite{at-palmer}.

Model pemilihan keputusan oleh kendaraan otonom perlu dilatih sehingga kendaraan
otonom tersebut dapat diandalkan. Oleh karena itu, diperlukan penelitian dan
percobaan lebih lanjut untuk memngembangkan algoritma, kemampuan evaluasi, dan
protokol kendaraan otonom. Penelitan dan eksperimen kendaraan otonom dapat
dilakukan dengan melakukan simulasi detil yang banyak. Simulasi-simulasi
tersebut bergantung kepada model yang ada di dunia nyata \parencite{av-berger}.

Simulasi kendaraan otonom akan memudahkan pelatihan dan proses validasi strategi
kemudi kendaraan otonom. Simulasi mencegah terjadinya kejadian yang berbahaya
dan tidak diinginkan terjadi seperti jika penelitian fisik dilakukan. Penelitian
kendaraan otonom di dunia nyata memiliki banyak rintangan. Penelitian fisik
membutuhkan biaya yang besar untuk membangun kendaraan otonom tersebut,
infrastuktur untuk pengetesan yang memadai, dan biaya lainnya. Selain
dibutuhkannya biaya, dibutuhkan juga sumber daya manusia untuk mengoperasikan
tes. Pengumpulan data juga membutuhkan waktu yang lama karena dibutuhkannya data
yang banyak, data yang valid, dan data yang mencakup kasus-kasus tidak terduga
untuk melatih kendaraan otonom \parencite{carla-dosovitskiy}.

% TODO:
% autonomous vehicles \subsection{Kendaraan Otonom di Indonesia}
% \subsection{Trem Otonom di Indonesia}
% \subsection{Lingkungan Berkendara di Indonesia ?}

% \section{Lingkungan Berkendara di Indonesia}
% street condition, layout, surrounding objects
% \section{Peraturan Lalu Lintas Indonesia}
% traffic rules in Indonesia

\section{Simulator CARLA}

\textit{Car Learning to Act} atau CARLA merupakan sebuah perangkat lunak
\textit{open-source} yang berguna untuk melakukan simulasi kendaraan otonom dan
lingkungan urban. CARLA telah dikembangkan untuk mendukung pelatihan, pembuatan
purwarupa, dan pengujian strategi kemudi otonom, baik dari sisi persepsi maupun
sisi pengontrolan. CARLA menyediakan model dasar untuk lingkungan simulasi,
sejumlah kendaraan, rambu lalu lintas, dan pejalan kaki. Selain itu, CARLA juga
menyediakan keluaran sensor-sensor dan sinyal-sinyal yang dapat digunakan untuk
melatih strategi kemudi seperti koordinat GPS, kecepatan, percepatan, data rinci
mengenai tabrakan yang terjadi pada model kendaraan, dan lain-lain.
Sensor-sensor dan kondisi lingkungan simulasi tersebut dapat diatur sesuai
dengan kebutuhan \parencite{carla-dosovitskiy}.

% \section{\textit{CARLA Simulation Engine}}
CARLA dikembangkan sedemikan rupa sehingga fleksibel untuk disesuaikan dengan
mudah dan juga realistis secara visual dan secara fisika. CARLA diimplementasi
sebagai layar \textit{open-source} di atas Unreal Engine 4 (UE4) sehingga dapat
dikembangkan lebih lanjut oleh komunitas \textit{open-source}.
\textit{Simulation engine} tersebut menyediakan kualitas \textit{render} yang
terkemuka, hukum fisika yang realistis, logika \textit{Non-Playable Character}
(NPC) dasar, dan banyak \textit{plugin} \parencite{carla-dosovitskiy}.

CARLA menyimulasikan sebuah dunia yang dinamis dan juga menyediakan antarmuka
sederhana antara dunia tersebut dan aktor atau agen yang berinteraksi dengan
dunia tersebut. CARLA dirancang sebagai sistem \textit{server-client} sehingga
fungsionalitas tersebut dapat direalisasikan. Server CARLA menjalankan dan
menyimulasikan suasana. Klien bertanggung jawab untuk mengatur interaksi antara
aktor atau agen dengan server via \textit{socket} dengan mengirimkan perintah
untuk aktor dan perintah untuk mengatur peraturan lingkungan simulasi di server.
Klien akan menerima umpan balik hasil rekaman sensor
\parencite{carla-dosovitskiy}.

Lingkungan simulasi pada CARLA terdiri atas model-model 3 dimensi (3D) yang
statik seperti bangungan, tumbuh-tumbuhan, rambu lalu lintas, dan infrastuktur
lainnya. Selain itu, terdapat juga model-model dinamis seperti kendaraan dan
pejalan kaki. Model-model 3D tersebut telah dibuat sehingga dimensi model-model
tersebut mencerminkan dimensi objek aslinya di dunia nyata. Model-model tersebut
dibuat menggunakan model geometrik yang ringan dan dengan tekstur yang sesuai
sehingga terlihat realistis dan detil serta dapat di-\textit{render} dengan
cepat. Aset-aset yang telah disediakan CARLA sangat dapat disesuaikan
\parencite{carla-dosovitskiy}. Aset baru juga dapat ditambahkan
\parencite{carla-documentation-intro}.

% TODO: ? add about how to import or make new assets
% TODO: ? istilah ? add about collision? bounding box blablabla

\section{Unreal Engine}

Unreal Engine merupakan sebuah \textit{game engine} yang dikembangkan oleh Epic
Games. Unreal Engine merupakan sebuah perangkat lunak yang dapat digunakan untuk
membuat permainan, simulasi, dan aplikasi multimedia lainnya. Unreal Engine
memiliki fitur-fitur untuk membuat konten tiga dimensi yang sangat lengkap
\parencite{ue-5}. Jika fitur atau ekosistem Unreal Engine kurang sesuai dengan
kebutuhan pengembang, Unreal Engine menyediakan \textit{source code} yang dapat
dimodifikasi \parencite{ue-4}. CARLA menggunakan Unreal Engine untuk
pengembangan lingkungan simulasi \parencite{carla-documentation-build}.

% TODO: maybe how to add objects?

\section{Blender}

Blender adalah sebuah perangkat lunak \textit{open-source} untuk pembuatan model
3D. Blender dapat digunakan di berbagai sistem operasi dan berjalan lancar pada
Linux, Windows, dan Macintosh. Blender menyediakan berbagai fitur yang mendukung
keseluruhan \textit{pipeline} model 3D. Fitur-fitur tersebut adalah pembuatan
model, \textit{rigging}, pembuatan animasi, pembuatan simulasi,
\textit{rendering}, pembuatan komposit, pelacakan gerakan (\textit{motion
tracking}), pengeditan video, dan pembuatan permainan. Blender menyediakan API
yang dapat digunakan untuk penyesuaian Blender sendiri dan pembuatan alat
pengeditan khusus. API Blender dapat digunakan menggunakan bahasa pemrograman
Python \parencite{blender-about}.

\subsection{Animasi dan \textit{Rigging}}

Animasi adalah membuat sebuah objek
bergerak atau berubah bentuk dari waktu ke waktu. \textit{Rigging} adalah
istilah untuk menambahkan kontrol ke sebuah objek dengan tujuan untuk
menganimasikan objek tersebut. \textit{Rigging} pada umumnya melibatkan komponen
\textit{armature} dan komponen \textit{constraints}. \textit{Armature} berguna
agar objek memiliki sendi yang fleksibel dan sering digunakan untuk animasi
skeletal. \textit{Constraints} adalah batasan dari gerakan objek.
\parencite{blender-animation-and-rigging}.

\subsection{\textit{Armature}}

\textit{Armature} pada Blender merupakan objek yang serupa dengan sistem
kerangka manusia (sistem skeletal) terutama fungsionalitasnya. \textit{Armature}
terdiri atas kumpulan tulang atau \textit{bone} yang memiliki struktur/hierarki
dan berfungsi untuk memberi gerakan ke sebuah objek/model 3D. \textit{Armature}
didesain untuk diberi pose. Gambar \ref{fig:basic-armature} menunjukkan contoh
struktur \textit{armature} sederhana pada Blender. Kumpulan \textit{bone} pada
sebuah \textit{armature} tidak perlu berhubungan satu sama lain sehingga
struktur dari \textit{armature} dapat berupa-rupa, misalnya seperti rantai
tulang atau \textit{chains of bones} yang dapat dilihat pada Gambar
\ref{fig:chains-of-bones} \parencite{blender-armature-introduction,
blender-armature-structure}. Setelah pembuatan \textit{armature} selesai, perlu
dilakukan \textit{skinning} agar pergerakan dari \textit{armature} berdampak
pada objek lain. Proses \textit{skinning} merupakan penghubungan antara objek
\textit{armature} dengan sebuah objek atau objek model/\textit{mesh}
\parencite{blender-skinning-introduction}. Gambar \ref{fig:basic-armature} juga
menunjukkan \textit{armature} yang telah di-\textit{skinning}.

% TODO: find a better way to set counter
\setcounter{section}{1}
\begin{figure}[ht]
    \centering
    \includegraphics[width=1.0\textwidth]{resources/chapter-2-basic-armature.png}
    \caption{\textit{Armature} \parencite{blender-armature-structure}}
    \label{fig:basic-armature}
\end{figure}

\begin{figure}[ht]
    \centering
    \includegraphics[width=1.0\textwidth]{resources/chapter-2-chain-of-bones.png}
    \caption{\textit{Armature} \parencite{blender-armature-structure}}
    \label{fig:chains-of-bones}
\end{figure}

\subsection{\textit{Bone}}

\textit{Bone} merupakan komponen dari \textit{armature}
\parencite{blender-bones-introduction}. Struktur sebuah \textit{bone} dapat
dilihat pada Gambar \ref{fig:bone-structure}. Sebuah \textit{bone} terdiri atas
3 subkomponen adalah sebagai berikut:
\parencite{blender-armature-structure,blender-glossary}:

\begin{enumerate}
    \item Sendi awal (\textit{root}/\textit{head})

    Sendi awal memiliki koordinat pada sumbu X, Y, dan Z di ruang lokal
    (\textit{local space}) objek \textit{armature}. Sendi awal merupakan titik
    acuan untuk rotasi \textit{bone} pada sumbu X dan Z pada (\textit{local
    space}). Rotasi ini disebut juga sebagai \textit{roll angle}.

    \item Sendi akhir (\textit{tip}/\textit{tail})

    Sendi akhir memiliki koordinat pada sumbu X, Y, dan Z relatif terhadap sendi
    awal.

    \item Badan (\textit{body})

    Badan oktahedral dari \textit{bone} terbentuk untuk menghubungkan sendi awal
    dan sendi akhir. Bagian sudut oktahedral yang lebih besar berhubungan
    dengan sendi awal. Sebaliknya, bagian sudut oktahedral yang lebih kecil
    berhubungan dengan sendi akhir. Badan \textit{bone} menentukan arah sumbu Y
    lokal dan rotasi \textit{bone} objek ketika diposekan.

\end{enumerate}

\begin{figure}[ht]
    \centering
    \includegraphics[width=1.0\textwidth]{resources/chapter-2-bone-structure.png}
    \caption{\textit{Armature} \parencite{blender-armature-structure}}
    \label{fig:bone-structure}
\end{figure}

% \section{RoadRunner}
% RoadRunner merupakan sebuah \textit{interactive editor} yang digunakan untuk
% membuat konten desain 3D untuk simulasi dan pengujian sistem kendaraan otonom.
% Pengguna dapat membuat desain jalanan dengan mengedit lingkungan dan menambahkan
% rambu lalu lintas, sinyal lalu lintas, persimpangan, dan batas kendaraan. Selain
% itu, RoadRunner dapat digunakan untuk membuat desain kota \parencite{roadrunner}.

\section{Penelitian Terkait}

Subbab ini membahas beberapa penelitian terkait dengan Tugas Akhir ini.
Penelitian-penelitian berikut menjadi rujukan dalam penelitian dan pengembangan
Tugas Akhir ini.

\subsection{Pengembangan Sistem Otonomi dengan Menggunakan Kecerdasan Artificial untuk Trem}

Penelitian oleh \cite{rispro-trilaksono} merupakan penelitian mengenai inovasi
bidang otomotif yang mengembangkan sistem otonomi untuk Trem menggunakan
kecerdasan buatan. Penelitian tersebut memiliki tiga tahap (dalam tiga tahun)
yaitu, pengembangan \textit{tram driving assistance}, pengembangan trem otonom,
dan pengujian trem otonom di \textit{mixed traffic} (lalu lintas yang melibatkan
berbagai pihak dan/atau kendaraan) serta persiapan komersialisasi.

Berikut Indikator Kinerja Riset (IKR)  atau luaran penelitian tersebut adalah
sebagai berikut \parencite{rispro-trilaksono}:

\begin{enumerate}

    \item Pengembangan algoritma persepsi untuk mengenali objek dan
    \textit{tracking} di lingkungan trem pada cuaca normal dan implementasi
    dalam bentuk \textit{software}.

    % Algoritma persepsi yang dikembangkan berupa \textit{panoptic segmentation},
    % \textit{camera object detection}, \textit{LIDAR object detection},
    % \textit{camera-LIDAR fusion object detection}, dan \textit{camera-radar
    % fusion object detection}

    \item Pengembangan algoritma perencanaan jalur untuk \textit{decision
    making} kendali kecepatan trem dan implementasi dalam bentuk
    \textit{software}.

    % Algoritma persepsi yang dikembangkan berupa \textit{railway estimator},
    % \textit{trajectory prediction}, dan \textit{safety assessment}

    \item Pengambilan \textit{dataset} lingkungan trem

    \textit{Dataset} yang diambil bentuk gambar citra kamera biasa, gambar citra
    LIDAR, dan data lainnya kemudian diolah dengan penambahan anotasi
    seperlunya.

    \item Pengembangan \textit{Driving Assistance System} untuk trem berupa:
    \textit{object detection & collision-avoidance assist},
    \textit{speed limit assist}, \textit{face recognition & driver
    attention warning}.

    \item Pengujian, analisis, dan desain ulang algoritma yang dikembangkan pada
    \textit{Software-in-the-Loop Simulation} (SILS) dan
    \textit{Hardware-in-the-Loop Simulation} (HILS).

    Menguji dan menganalisis \textit{platform} simulasi yaitu simulator CARLA
    versi 0.9.12, uji coba beragam semsor, dan uji coba beberapa algoritma yang
    sudah dikembangkan. Melakukan konversi kode program/algoritma yang telah
    dikembangkan ke bahasa pemrograman C++ untuk dimasukkan ke NVIDIA drive
    Pegasus dan membuat \textit{web-service} untuk menghubungkan server simulasi
    (SILS) dengan NVIDIA drive Pegasus (HILS). NVIDIA drive Pegasus merupakan
    \textit{hardware} untuk memroses algoritma  \textit{adaptive cruise control}
    (ACC), \textit{emergency braking system} (EBS), dan \textit{collision
    avoidance}. Dari pengujian dan analisis tersebut, dibutuhkan penyempurnaan
    algoritma persepsi berbasis sensor, perbaikan arsitekstur (HILS), dan
    integrasi objek lokal pada skenario simulasi.

    \item Pengembangan dan manufaktur \textit{platform} trem, sistem
    \textit{drive-by-wire} pada trem, dan integrasi sensor.

    \item Publikasi ilmiah.
    \item Draf kekayaan intelektual.
    \item \textit{Self-assessment} Tingkat Kesiapan Teknologi.
    \item Penyusunan poster ilmiah populer atas pelaksanaan dan hasil riset.

\end{enumerate}

Penelitian tersebut sedang dalam tahap kedua yaitu pengembangan trem otonom.
Masing-masing indikator telah atau sedang dalam perkembangan
\parencite{rispro-trilaksono}. Tugas Akhir \textit{Capstone} ini merupakan
bagian dari penelitian tersebut, khususnya pada indikator mengenai pengembangan
SILS dan HILS.

\subsection{\textit{KIT Bus: A Shuttle Model for CARLA Simulator}}

Penelitian \textit{KIT Bus: A Shuttle Model for CARLA Simulator} ini merupakan
penelitian membuat model \textit{shuttle bus} untuk CARLA yang dilakukan oleh
\cite{related-work-xiang}. Proses pembuatan model tersebut dilakukan dalam tiga
tahap, yaitu sebagai berikut \parencite{related-work-xiang}:

\begin{enumerate}

    \item Membuat model 3D dari bus menggunakan aplikasi 3ds Max.

    Model bus dibuat sesuai dengan referensi dan dimensi asli. Model bus yang
    dibuat secara detil dan berpermukaan mulus sehingga realistis. Model bus
    terdiri atas 6 bagian yaitu, bagian badan, bagian roda-roda, bagian
    interior, bagian detil, bagian kaca, dan bagian plat kendaraan.

    \item Melakukan pengeditan model 3D bus dalam CARLAUE4.

    Model bus yang telah dibuat kemudian diimpor ke dalam CARLAUE4.
    Masing-masing bagian badan bus dan keempat roda bus dipasangkan sebuah
    \textit{collision box} yang berbentuk dan berukuran sama. Agar model bus
    dapat menyimulasikan animasi bus yang sebenarnya, cetak biru animasi harus
    dibuat berbagai macam bagian bus yang bergerak. Dilakukan juga penambahan
    material atau tekstur dan fungsionalitas lainnya yang dibutuhkan seperti,
    lampu dan tekstur-tekstur yang ada pada permukaan bus. Hal tersebut
    dilakukan sehingga model bus dapat menyerupai model bus yang sebenarnya.
    Model bus yang telah selesai diedit dimasukkan ke dalam \textit{vehicle
    factory} agar dapat bisa dimunculkan (\textit{spawn}).

    \item Memverifikasi kontrol manual dan otonom dari model 3D bus yang telah
    dibangun dalam CARLA.

    Model bus yang telah dibuat kemudian diuji untuk memverifikasi apakah model
    bus tersebut dapat dioperasikan secara manual dan otonom. Model bus dites
    untuk berjalan maju, mengerem untuk melambat, mengemudi ke kiri dan ke
    kanan, dan memindahkan gigi kopling.

\end{enumerate}

\subsection{\textit{The Autonomous Siemens Tram}}

Penelitian \textit{The Autonomous Siemens Tram} membahas trem otonom Siemens
yang telah didemonstrasikan di Potsdam, Jerman pada tahun 2018. Sistem trem
otonom tersebut dibangun di atas trem Siemens Combino dan menggunakan
sensor-sensor multi-modal untuk mengidentifikasi lokasi kendaraan serta
mendeteksi dan merespon lampu lalu lintas dan objek lain. Trem otonom memiliki
beberapa komputer yang memiliki \textit{Graphics Processing Unit} (GPU) yang
memadai untuk memroses data dari sensor LIDAR, sensor radar, kamera pendeteksi
objek, dan kamera pendekteksi sinyal. Trem otonom tersebut dapat beroperasi
dengan lancar namun memiliki kendala ketika melakukan lokalisasi trem hanya
dengan Global Navigation Satellite System (GNSS). Penelitian masih dilakukan
untuk mengatasi kendala tersebut, misalnya dengan melakukan lokalisasi dengan
persepsi atau visual \parencite{at-palmer}.

% \blankpage
\chapter{Deskripsi Solusi}
\label{chapter-3}

Bab ini membahas deskripsi umum permasalahan \textit{capstone}, analisis masalah
simulasi yang sudah ada, analisis terhadap objek dan lingkungan yang akan
diimplementasi, analisis solusi yang akan diterapkan untuk mengatasi masalah.

\section{Deskripsi Umum Permasalahan \textit{Capstone}}

Proyek trem otonom menggunakan simulasi untuk mempercepat dan mempermudah
pengujian dan validasi pengembangan model/algoritma \textit{decision making},
persepsi, \textit{localization}, dan \textit{mapping} trem otonom. Tim
\textit{capstone} Tugas Akhir ini merupakan bagian dari tim simulasi yang
bertugas untuk mengembangkan simulasi yang sudah ada. Kemajuan
pengembangan simulasi atau pengembangan SILS dan HILS adalah sebagai berikut:

\begin{enumerate}

	\item Eksplorasi kakas CARLA sebagai simulator.

	CARLA versi 0.9.12 diinstal. Model angkot dan becak telah diimpor
	sebagai objek statis (bukan kendaraan). Kendaraan model \textit{FireTruck}
	digunakan sebagai \textit{ego vehicle} menggantikan trem untuk sementara
	waktu karena model trem belum diintegrasi sebagai kendaraan.

	\item \textit{Web service} telah diimplementasi sebagai jalur komunikasi
	antara simulator CARLA dan Pegasus pada HILS.

	Komunikasi menggunakan \textit{web service} lambat karena jumlah transaksi
	per detik kecil dan simulasi juga lambat. Eksplorasi CARLA ROS Bridge
	dilakukan namun belum selesai.

	\item Menguji sensor virtual, mencoba menjalankan algoritma \textit{decision
	making} dan persepsi.

	Sensor virtual diuji apakah layak untuk digunakan sebagai pengganti sensor
	asli. Algoritma \textit{decision making} dan persepsi diuji juga pada HILS.

	\item Membuat rancangan skenario simulasi.

	Rancangan daftar skenario simulasi untuk menguji berbagai skenario dengan
	variabel cuaca, waktu, kecepatan trem, dan lalu lintas yang berbeda.

\end{enumerate}

Dari kemajuan proyek dari tahun sebelumnya dibutuhkan  penyempurnaan algoritma
persepsi berbasis sensor, perbaikan arsitektur HILS atau komunikasi yang
memadai, dan integrasi objek lokal pada skenario simulasi. Tim \textit{capstone}
yang beranggotakan 3 orang bertanggung jawab untuk:

\begin{enumerate}

	\item Mengembangkan mekanisme komunikasi antarperangkat dalam arsitektur
	HILS yang lebih lancar dan lebih cepat.
	\item Membuat implementasi skenario pengujian simulasi.
	\item Mengintegrasikan dan/atau mengimplementasikan objek lokal ke skenario
	simulasi.

\end{enumerate}

Tugas Akhir ini bertujuan untuk mengimplementasikan objek trem, objek lokal
lingkungan Indonesia di simulasi menggunakan CARLA agar lingkungan simulasi
menyerupai lingkungan aslinya sehingga mendukung pengembangan pengujian dan
validasi \textit{decision making} dan persepsi dengan menggunakan simulasi.

\section{Analisis Masalah Simulasi dan Aset Simulasi}

Simulasi trem otonom untuk saat ini telah diinisiasi namun baru sebatas
eksplorasi simulator CARLA dengan menambahkan kendaraan angkot dan becak sebagai
objek statis (bukan kendaraan) dan mengetes data dari sensor di \textit{ego
vehicle} dalam simulasi. Aset model 3D sudah dibangun namun belum
diimplementasikan ke dalam simulasi. Aset model 3D tersebut adalah: trem,
angkot, becak, sepeda onthel, sepeda motor, beberapa rambu lalu lintas, dan
gerobak. Gambar \ref{fig:3d-model-assets} menunjukkan aset model 3D yang telah
dibuat.

\begin{figure}[!tb]
% \begin{figure}[ht]
	\centering
	\subfloat[Trem]{\includegraphics[width=0.4\textwidth]{resources/chapter-3-tram.png}}
	\hfill
	\subfloat[Angkot]{\includegraphics[width=0.4\textwidth]{resources/chapter-3-angkot.png}}
	\hfill
	\subfloat[Becak]{\includegraphics[width=0.3\textwidth]{resources/chapter-3-becak.png}}
	\hfill
	\subfloat[Sepeda onthel]{\includegraphics[width=0.3\textwidth]{resources/chapter-3-sepeda-onthel.png}}
	\hfill
	\subfloat[Sepeda motor]{\includegraphics[width=0.3\textwidth]{resources/chapter-3-sepeda-motor.png}}
	\hfill
	\subfloat[Rambu-rambu lalu lintas]{\includegraphics[width=0.4\textwidth]{resources/chapter-3-rambu.png}}
	\hfill
	\subfloat[Gerobak]{\includegraphics[width=0.4\textwidth]{resources/chapter-3-gerobak.png}}
	\caption{Aset model 3D \parencite{rispro-trilaksono}}
	\label{fig:3d-model-assets}
\end{figure}

Selain aset angkot dan becak yang diimpor sebagai objek statis, aset simulasi
untuk saat ini baru berupa aset bawaan dari CARLA. Aset bawaan CARLA seperti
peta kota, kendaraan, bangunan, rambu lalu lintas, dan lain-lain merupakan aset
yang mencerminkan kota-kota di Amerika Serikat pada umumnya. Implementasi objek
dan lingkungan Indonesia dibutuhkan agar simulasi serupa dengan kehidupan nyata
di Indonesia sehingga ketika pengujian strategi kemudi trem otonom memiliki
tingkat akurasi dan presisi yang tinggi. Simulasi yang sesuai dengan keadaan
aslinya sangat berpengaruh terhadap sistem trem otonom. Diperlukan penambahan
aset model 3D lain yang juga perlu diimplementasikan pada simulasi. Aset
tersebut adalah stasiun trem, rel trem, dan rambu yang lengkap.

Modul simulasi ini merupakan bagian dari pengembangan sistem otonom dengan
menggunakan kecerdasan buatan untuk trem. Modul ini bertujuan untuk memenuhi
kebutuhan pengujian virtual strategi kemudi trem otonom dengan simulasi.
Simulasi ini bertujuan untuk melakukan validasi strategi kemudi kecerdasan
buatan trem otonom yang telah dikembangkan. Validasi diperlukan agar kecerdasan
buatan yang dikembangkan dapat beroperasi dengan lancar di lingkungan Indonesia.
Oleh karena itu, dibutuhkan aset simulasi yang sesuai dengan keadaan Indonesia.

\section{Analisis Solusi}

Permasalahan yang telah dibahas dapat diselesaikan dengan mengedit aset yang
sudah ada dan menambahkan aset baru. Aset baru yang lain dapat dibuat
menggunakan aplikasi \textit{3D modelling}. Aset dibuat secara lengkap dan
detail agar perilaku aset sesuai dengan yang asli.

% the sentence below goes after the first sentence
% Aset seperti peta kota harus dibuat langsung menggunakan editor CARLAUE4 atau
% menggunakan RoadRunner kemudian diimpor ke dalam editor CARLAUE4 jika aset
% peta diperlukan.

Aset model 3D yang telah jadi selanjutnya diimpor ke dalam editor CARLAUE4.
Proses impor aset ini dilakukan dengan mengikuti panduan yang telah tersedia di
dokumentasi CARLA. Proses impor aset harus dilakukan dengan benar agar perilaku
aset baik dan stabil untuk simulasi. Terdapat aset khusus yang harus ditambahkan
ke dalam berkas 3D model kendaraan dan diatur ke model kendaraan. Aset khusus
tersebut merupakan aset \textit{armature} untuk \textit{rigging} roda kendaraan.
Setelah pengaturan aset dalam editor CARLAUE4 selesai, dilakukan verifikasi aset
dengan cara menjalankan simulasi dan mengamati cara kendaraan beroperasi.
Implementasi aset berupa kendaraan bus telah berhasil dilakukan pada penelitian
lain yang dibahas pada Subbab \ref{subsec:kitbus}. Proses implementasi kendaraan
bus dilakukan dengan membuat model 3D bus, mengimpor model 3D bus ke dalam
editor CARLAUE4, mengedit aset bus tersebut dalam editor CARLAUE4, dan
memverifikasi hasil impor aset bus tersebut.

% Implementasi objek dan lingkungan Indonesia yang akan dilakukan adalah dengan
% menambahkan/mengimpor aset, membuat aset baru, dan mengedit aset yang sudah ada.
% Implementasi yang menambahkan aset meliputi kendaraan, rambu lalu lintas, rel,
% dan stasiun. Kendaraan yang akan diimplementasikan adalah trem otonom, angkot,
% becak, motor, dan sepeda onthel.
% Trem akan dipasangkan berbagai sensor, seperti LIDAR, radar, kamera RGB untuk
% menangkap lingkungan sekitar.

% note: gerobak is not going to be implemented

Penambahan atau pengeditan aset CARLA membutuhkan simulator CARLA, CARLAUE4, dan
editor CARLAUE4 yang dibangun atau di-\textit{compile} sendiri bukan menggunakan
program, \textit{binary}, atau yang sudah \textit{packaged}. Dibutuhkan aplikasi
\textit{3D modelling} seperti Blender untuk membuat aset model 3D.
% Aplikasi RoadRunner akan digunakan untuk membuat aset peta.

\section{Rancangan Implementasi Objek dan Lingkungan Indonesia di Simulator CARLA}

Proses implementasi objek dan lingkungan di simulator CARLA dilakukan dengan
langkah-langkah sebagai berikut:

\begin{enumerate}

	\item Eksplorasi editor CARLAUE4 versi 0.9.12 dan versi 0.9.13.

	Eksplorasi dilakukan untuk memelajari cara menggunakan editor CARLAUE4 dan
	mengetahui fitur-fitur yang telah dikembangkan.

	\item Membuat aset baru dengan Blender.

	Aset baru yang dibutuhkan dibuat mengikuti dokumentasi CARLA mengenai
	penambahan aset-aset yang bersangkutan. Aset baru yang butuh
	diimplementasikan adalah stasiun trem, rel trem, dan rambu lalu lintas.
	% Hal tersebut meliputi bentuk poligon, jumlah poligon, jumlah VEF, dan
	% lain-lain. (yang sesuai dengan petunjuk)

	\item Melakukan impor dan edit aset ke dalam editor CARLAUE4.

	Aset yang ingin dimasukkan ke lingkungan simulasi dan diedit harus dilakukan
	dengan aplikasi CARLAUE4.

	% aset carla berbentuk .uasset yang merupakan binary file

	% untuk import vehicles:
	% https://carla.readthedocs.io/en/0.9.13/tuto_A_add_vehicle/#bind-and-model-the-vehicle
	% ~~untuk buat peta:~~
	% ~~https://carla.readthedocs.io/en/0.9.13/tuto_M_generate_map/~~

	% links:
	% https://carla.readthedocs.io/en/0.9.13/tuto_A_add_props/
	% https://carla.readthedocs.io/en/0.9.13/tuto_A_material_customization/

	\item Melakukan penambahan dan validasi aset yang telah diimpor.

	Aset yang telah diimpor dan diedit dimasukkan ke dalam lingkungan simulasi
	(\textit{world}). Validasi aset dapat dilakukan dengan menjalankan simulator
	dan mengamati aset dalam simulasi. Validasi ini dilakukan untuk memastikan
	aset yang ditambahkan sudah sesuai dan stabil dalam simulasi.

	\item Melakukan ekspor hasil implementasi agar dapat didistribusikan.

	Aset yang ingin digunakan oleh pengguna lain dapat diekspor dalam bentuk
	versi \textit{packaged} sehingga \textit{portable} dan ringan ketika
	dijalankan. Hal ini dapat dilakukan dengan menjalankan perintah di terminal.

\end{enumerate}

\chapter{Implementasi dan Evaluasi}

\section{Implementasi}
\blindtext

\section{Tujuan Pengujian}
\blindtext

\section{Skenario Pengujian}
\blindtext

\section{Hasil Pengujian}
\blindtext

\section{Evaluasi}
\blindtext

\chapter{Kesimpulan dan Saran}

\section{Kesimpulan}
% TODO: write conclusion
% important things: bone gaboleh diskala
\blindtext

\section{Saran}
% TODO: write suggestions
\blindtext

%----------------------------------------------------------------%

% Daftar pustaka
\printbibliography
% \blankpage

% Setting judul lampiran
\titlespacing*{\chapter}{0pt}{0pt}{0pt}
\titlespacing*{\section}{0pt}{0pt}{*1}

% Setting judul anak lampiran
\titleformat*{\section}{\bfseries}

% Index
\appendix
% \chapter{Rancangan Sistem}\label{appendix-arsitektur-baru}
% note: find a better way to set counter
\setcounter{section}{1}
\begin{figure}[ht]
	\centering
	\includegraphics[width=1.0\textwidth]{resources/appendix-1-deployment diagram.png}
	\caption{Rancangan Sistem Simulasi}
\end{figure}

% \input{chapters/appendix-2.tex}

\end{document}
